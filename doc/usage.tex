%!TEX root = std.tex
\rSec0[atomicc.usage]{Usage}

\rSec1[atomicc.command]{Command line switches}
Command line switches...

\rSec1[atomicc.printf]{debugging with printf}

To aid debugging with a simulator, "printf" statements in __module declarations are
translated to "\$display" statements in the generated verilog.
For debugging with synthesized hardware, "printf" statements are translated into
indication packets sent through the NOC back to the software side host program.
The format strings for the printf statements are placed into a generated file
in generated/xxx.generated.printf along with a list of the bit lengths for each
parameter to the printf.

\begin{bnf}
\nontermdef{printf-declaration}\br
    \terminal{__printf ;}
\end{bnf}

To use the NOC printf:
\begin{itemize}
\item add the following line to the __module being tested:
\begin{codeblock}
     __printf;
\end{codeblock}
\item add a line similar to the following (with the 'xxx' replaced) to the test program:
\begin{codeblock}
     atomiccPrintfInit("generated/rulec.generated.printf");
\end{codeblock}
\end{itemize}
