\infannex{gram}{Grammar integration with C++ summary}

\pnum
Summary of C++ grammar

\rSec1[gram.key]{Keywords}

\pnum
\indextext{keyword}%
New context-dependent keywords are introduced into a program by
\tcode{typedef}\iref{dcl.typedef},
\tcode{namespace}\iref{namespace.def},
class\iref{class}, enumeration\iref{dcl.enum}, and
\tcode{template}\iref{temp}
declarations.

\begin{ncbnf}
typedef-name:\br
	identifier
\end{ncbnf}

\begin{ncbnf}
namespace-name:\br
	identifier\br
	namespace-alias

namespace-alias:\br
	identifier
\end{ncbnf}

\begin{ncbnf}
class-name:\br
	identifier\br
	simple-template-id
\end{ncbnf}

\begin{ncbnf}
enum-name:\br
	identifier
\end{ncbnf}

\begin{ncbnf}
template-name:\br
	identifier
\end{ncbnf}

Note that a
\grammarterm{typedef-name}
naming a class is also a
\grammarterm{class-name}\iref{class.name}.


\gramSec[gram.lex]{Lexical conventions}

\begin{bnf}
\nontermdef{hex-quad}\br
    hexadecimal-digit hexadecimal-digit hexadecimal-digit hexadecimal-digit
\end{bnf}

\begin{bnf}
\nontermdef{universal-character-name}\br
    \terminal{\textbackslash u} hex-quad\br
    \terminal{\textbackslash U} hex-quad hex-quad
\end{bnf}

\begin{bnf}
\nontermdef{preprocessing-token}\br
    header-name\br
    identifier\br
    pp-number\br
    character-literal\br
    user-defined-character-literal\br
    string-literal\br
    user-defined-string-literal\br
    preprocessing-op-or-punc\br
    \textnormal{each non-white-space character that cannot be one of the above}
\end{bnf}

\begin{bnf}
\nontermdef{token}\br
    identifier\br
    keyword\br
    literal\br
    operator\br
    punctuator
\end{bnf}

\begin{bnf}
\nontermdef{header-name}\br
    \terminal{<} h-char-sequence \terminal{>}\br
    \terminal{"} q-char-sequence \terminal{"}
\end{bnf}

\begin{bnf}
\nontermdef{h-char-sequence}\br
    h-char\br
    h-char-sequence h-char
\end{bnf}

\begin{bnf}
\nontermdef{h-char}\br
    \textnormal{any member of the source character set except new-line and \terminal{>}}
\end{bnf}

\begin{bnf}
\nontermdef{q-char-sequence}\br
    q-char\br
    q-char-sequence q-char
\end{bnf}

\begin{bnf}
\nontermdef{q-char}\br
    \textnormal{any member of the source character set except new-line and \terminal{"}}
\end{bnf}

\begin{bnf}
\nontermdef{pp-number}\br
    digit\br
    \terminal{.} digit\br
    pp-number digit\br
    pp-number identifier-nondigit\br
    pp-number \terminal{'} digit\br
    pp-number \terminal{'} nondigit\br
    pp-number \terminal{e} sign\br
    pp-number \terminal{E} sign\br
    pp-number \terminal{p} sign\br
    pp-number \terminal{P} sign\br
    pp-number \terminal{.}
\end{bnf}

\begin{bnf}
\nontermdef{identifier}\br
    identifier-nondigit\br
    identifier identifier-nondigit\br
    identifier digit
\end{bnf}

\begin{bnf}
\nontermdef{identifier-nondigit}\br
    nondigit\br
    universal-character-name
\end{bnf}

\begin{bnf}
\nontermdef{nondigit} \textnormal{one of}\br
    \terminal{a b c d e f g h i j k l m}\br
    \terminal{n o p q r s t u v w x y z}\br
    \terminal{A B C D E F G H I J K L M}\br
    \terminal{N O P Q R S T U V W X Y Z _}
\end{bnf}

\begin{bnf}
\nontermdef{digit} \textnormal{one of}\br
    \terminal{0 1 2 3 4 5 6 7 8 9}
\end{bnf}

\begin{bnf}
%% Ed. note: character protrusion would misalign various operators.
\microtypesetup{protrusion=false}\obeyspaces
\nontermdef{preprocessing-op-or-punc} \textnormal{one of}\br
    \terminal{\{        \}        [        ]        \#        \#\#       (        )}\br
    \terminal{<:       :>       <\%       \%>       \%:       \%:\%:     ;        :        ...}\br
    \terminal{new      delete   ?        ::       .        .*       ->       ->*      \~}\br
    \terminal{!        +        -        *        /        \%        \caret{}        \&        |}\br
    \terminal{=        +=       -=       *=       /=       \%=       \caret{}=       \&=       |=}\br
    \terminal{==       !=       <        >        <=       >=       <=>      \&\&       ||}\br
    \terminal{<<       >>       <<=      >>=      ++       --       ,}\br
    \terminal{and      or       xor      not      bitand   bitor    compl}\br
    \terminal{and_eq   or_eq    xor_eq   not_eq}
\end{bnf}

\begin{bnf}
\nontermdef{literal}\br
    integer-literal\br
    character-literal\br
    floating-literal\br
    string-literal\br
    boolean-literal\br
    pointer-literal\br
    user-defined-literal
\end{bnf}

\begin{bnf}
\nontermdef{integer-literal}\br
    binary-literal \opt{integer-suffix}\br
    octal-literal \opt{integer-suffix}\br
    decimal-literal \opt{integer-suffix}\br
    hexadecimal-literal \opt{integer-suffix}
\end{bnf}

\begin{bnf}
\nontermdef{binary-literal}\br
    \terminal{0b} binary-digit\br
    \terminal{0B} binary-digit\br
    binary-literal \opt{\terminal{'}} binary-digit
\end{bnf}

\begin{bnf}
\nontermdef{octal-literal}\br
    \terminal{0}\br
    octal-literal \opt{\terminal{'}} octal-digit
\end{bnf}

\begin{bnf}
\nontermdef{decimal-literal}\br
    nonzero-digit\br
    decimal-literal \opt{\terminal{'}} digit
\end{bnf}

\begin{bnf}
\nontermdef{hexadecimal-literal}\br
    hexadecimal-prefix hexadecimal-digit-sequence
\end{bnf}

\begin{bnf}
\nontermdef{binary-digit} \textnormal{one of}\br
    \terminal{0  1}
\end{bnf}

\begin{bnf}
\nontermdef{octal-digit} \textnormal{one of}\br
    \terminal{0  1  2  3  4  5  6  7}
\end{bnf}

\begin{bnf}
\nontermdef{nonzero-digit} \textnormal{one of}\br
    \terminal{1  2  3  4  5  6  7  8  9}
\end{bnf}

\begin{bnf}
\nontermdef{hexadecimal-prefix} \textnormal{one of}\br
    \terminal{0x  0X}
\end{bnf}

\begin{bnf}
\nontermdef{hexadecimal-digit-sequence}\br
    hexadecimal-digit\br
    hexadecimal-digit-sequence \opt{\terminal{'}} hexadecimal-digit
\end{bnf}

\begin{bnf}
\nontermdef{hexadecimal-digit} \textnormal{one of}\br
    \terminal{0  1  2  3  4  5  6  7  8  9}\br
    \terminal{a  b  c  d  e  f}\br
    \terminal{A  B  C  D  E  F}
\end{bnf}

\begin{bnf}
\nontermdef{integer-suffix}\br
    unsigned-suffix \opt{long-suffix} \br
    unsigned-suffix \opt{long-long-suffix} \br
    long-suffix \opt{unsigned-suffix} \br
    long-long-suffix \opt{unsigned-suffix}
\end{bnf}

\begin{bnf}
\nontermdef{unsigned-suffix} \textnormal{one of}\br
    \terminal{u  U}
\end{bnf}

\begin{bnf}
\nontermdef{long-suffix} \textnormal{one of}\br
    \terminal{l  L}
\end{bnf}

\begin{bnf}
\nontermdef{long-long-suffix} \textnormal{one of}\br
    \terminal{ll  LL}
\end{bnf}

\begin{bnf}
\nontermdef{character-literal}\br
    \opt{encoding-prefix} \terminal{'} c-char-sequence \terminal{'}
\end{bnf}

\begin{bnf}
\nontermdef{encoding-prefix} \textnormal{one of}\br
    \terminal{u8}\quad\terminal{u}\quad\terminal{U}\quad\terminal{L}
\end{bnf}

\begin{bnf}
\nontermdef{c-char-sequence}\br
    c-char\br
    c-char-sequence c-char
\end{bnf}

\begin{bnf}
\nontermdef{c-char}\br
    \textnormal{any member of the source character set except the single-quote \terminal{'}, backslash \terminal{\textbackslash}, or new-line character}\br
    escape-sequence\br
    universal-character-name
\end{bnf}

\begin{bnf}
\nontermdef{escape-sequence}\br
    simple-escape-sequence\br
    octal-escape-sequence\br
    hexadecimal-escape-sequence
\end{bnf}

\begin{bnf}
\nontermdef{simple-escape-sequence} \textnormal{one of}\br
    \terminal{\textbackslash'}\quad\terminal{\textbackslash"}\quad\terminal{\textbackslash ?}\quad\terminal{\textbackslash\textbackslash}\br
    \terminal{\textbackslash a}\quad\terminal{\textbackslash b}\quad\terminal{\textbackslash f}\quad\terminal{\textbackslash n}\quad\terminal{\textbackslash r}\quad\terminal{\textbackslash t}\quad\terminal{\textbackslash v}
\end{bnf}

\begin{bnf}
\nontermdef{octal-escape-sequence}\br
    \terminal{\textbackslash} octal-digit\br
    \terminal{\textbackslash} octal-digit octal-digit\br
    \terminal{\textbackslash} octal-digit octal-digit octal-digit
\end{bnf}

\begin{bnf}
\nontermdef{hexadecimal-escape-sequence}\br
    \terminal{\textbackslash x} hexadecimal-digit\br
    hexadecimal-escape-sequence hexadecimal-digit
\end{bnf}

\begin{bnf}
\nontermdef{floating-literal}\br
    decimal-floating-literal\br
    hexadecimal-floating-literal
\end{bnf}

\begin{bnf}
\nontermdef{decimal-floating-literal}\br
    fractional-constant \opt{exponent-part} \opt{floating-suffix}\br
    digit-sequence exponent-part \opt{floating-suffix}
\end{bnf}

\begin{bnf}
\nontermdef{hexadecimal-floating-literal}\br
    hexadecimal-prefix hexadecimal-fractional-constant binary-exponent-part \opt{floating-suffix}\br
    hexadecimal-prefix hexadecimal-digit-sequence binary-exponent-part \opt{floating-suffix}
\end{bnf}

\begin{bnf}
\nontermdef{fractional-constant}\br
    \opt{digit-sequence} \terminal{.} digit-sequence\br
    digit-sequence \terminal{.}
\end{bnf}

\begin{bnf}
\nontermdef{hexadecimal-fractional-constant}\br
    \opt{hexadecimal-digit-sequence} \terminal{.} hexadecimal-digit-sequence\br
    hexadecimal-digit-sequence \terminal{.}
\end{bnf}

\begin{bnf}
\nontermdef{exponent-part}\br
    \terminal{e} \opt{sign} digit-sequence\br
    \terminal{E} \opt{sign} digit-sequence
\end{bnf}

\begin{bnf}
\nontermdef{binary-exponent-part}\br
    \terminal{p} \opt{sign} digit-sequence\br
    \terminal{P} \opt{sign} digit-sequence
\end{bnf}

\begin{bnf}
\nontermdef{sign} \textnormal{one of}\br
    \terminal{+  -}
\end{bnf}

\begin{bnf}
\nontermdef{digit-sequence}\br
    digit\br
    digit-sequence \opt{\terminal{'}} digit
\end{bnf}

\begin{bnf}
\nontermdef{floating-suffix} \textnormal{one of}\br
    \terminal{f  l  F  L}
\end{bnf}

\begin{bnf}
\nontermdef{string-literal}\br
    \opt{encoding-prefix} \terminal{"} \opt{s-char-sequence} \terminal{"}\br
    \opt{encoding-prefix} \terminal{R} raw-string
\end{bnf}

\begin{bnf}
\nontermdef{s-char-sequence}\br
    s-char\br
    s-char-sequence s-char
\end{bnf}

\begin{bnf}
\nontermdef{s-char}\br
    \textnormal{any member of the source character set except the double-quote \terminal{"}, backslash \terminal{\textbackslash}, or new-line character}\br
    escape-sequence\br
    universal-character-name
\end{bnf}

\begin{bnf}
\nontermdef{raw-string}\br
    \terminal{"} \opt{d-char-sequence} \terminal{(} \opt{r-char-sequence} \terminal{)} \opt{d-char-sequence} \terminal{"}
\end{bnf}

\begin{bnf}
\nontermdef{r-char-sequence}\br
    r-char\br
    r-char-sequence r-char
\end{bnf}

\begin{bnf}
\nontermdef{r-char}\br
    \textnormal{any member of the source character set, except a right parenthesis \terminal{)} followed by}\br
    \bnfindent\textnormal{the initial \grammarterm{d-char-sequence} (which may be empty) followed by a double quote \terminal{"}.}
\end{bnf}

\begin{bnf}
\nontermdef{d-char-sequence}\br
    d-char\br
    d-char-sequence d-char
\end{bnf}

\begin{bnf}
\nontermdef{d-char}\br
    \textnormal{any member of the basic source character set except:}\br
    \bnfindent\textnormal{space, the left parenthesis \terminal{(}, the right parenthesis \terminal{)}, the backslash \terminal{\textbackslash}, and the control characters}\br
    \bnfindent\textnormal{representing horizontal tab, vertical tab, form feed, and newline.}
\end{bnf}

\begin{bnf}
\nontermdef{boolean-literal}\br
    \terminal{false}\br
    \terminal{true}
\end{bnf}

\begin{bnf}
\nontermdef{pointer-literal}\br
    \terminal{nullptr}
\end{bnf}

\begin{bnf}
\nontermdef{user-defined-literal}\br
    user-defined-integer-literal\br
    user-defined-floating-literal\br
    user-defined-string-literal\br
    user-defined-character-literal
\end{bnf}

\begin{bnf}
\nontermdef{user-defined-integer-literal}\br
    decimal-literal ud-suffix\br
    octal-literal ud-suffix\br
    hexadecimal-literal ud-suffix\br
    binary-literal ud-suffix
\end{bnf}

\begin{bnf}
\nontermdef{user-defined-floating-literal}\br
    fractional-constant \opt{exponent-part} ud-suffix\br
    digit-sequence exponent-part ud-suffix\br
    hexadecimal-prefix hexadecimal-fractional-constant binary-exponent-part ud-suffix\br
    hexadecimal-prefix hexadecimal-digit-sequence binary-exponent-part ud-suffix
\end{bnf}

\begin{bnf}
\nontermdef{user-defined-string-literal}\br
    string-literal ud-suffix
\end{bnf}

\begin{bnf}
\nontermdef{user-defined-character-literal}\br
    character-literal ud-suffix
\end{bnf}

\begin{bnf}
\nontermdef{ud-suffix}\br
    identifier
\end{bnf}

\gramSec[gram.basic]{Basics}

\begin{bnf}
\nontermdef{translation-unit}\br
    \opt{declaration-seq}
\end{bnf}

\gramSec[gram.expr]{Expressions}

\begin{bnf}
\nontermdef{primary-expression}\br
    literal\br
    \terminal{this}\br
    \terminal{(} expression \terminal{)}\br
    id-expression\br
    lambda-expression\br
    fold-expression\br
    requires-expression
\end{bnf}

\begin{bnf}
\nontermdef{id-expression}\br
    unqualified-id\br
    qualified-id
\end{bnf}

\begin{bnf}
\nontermdef{unqualified-id}\br
    identifier\br
    operator-function-id\br
    conversion-function-id\br
    literal-operator-id\br
    \terminal{\~} class-name\br
    \terminal{\~} decltype-specifier\br
    template-id
\end{bnf}

\begin{bnf}
\nontermdef{qualified-id}\br
    nested-name-specifier \opt{\terminal{template}} unqualified-id
\end{bnf}

\begin{bnf}
\nontermdef{nested-name-specifier}\br
    \terminal{::}\br
    type-name \terminal{::}\br
    namespace-name \terminal{::}\br
    decltype-specifier \terminal{::}\br
    nested-name-specifier identifier \terminal{::}\br
    nested-name-specifier \opt{\terminal{template}} simple-template-id \terminal{::}
\end{bnf}

\begin{bnf}
\nontermdef{lambda-expression}\br
    lambda-introducer compound-statement\br
    lambda-introducer lambda-declarator \opt{requires-clause} compound-statement\br
    lambda-introducer \terminal{<} template-parameter-list \terminal{>} \opt{requires-clause} compound-statement\br
    lambda-introducer \terminal{<} template-parameter-list \terminal{>} \opt{requires-clause}\br
    \bnfindent lambda-declarator \opt{requires-clause} compound-statement
\end{bnf}

\begin{bnf}
\nontermdef{lambda-introducer}\br
    \terminal{[} \opt{lambda-capture} \terminal{]}
\end{bnf}

\begin{bnf}
\nontermdef{lambda-declarator}\br
    \terminal{(} parameter-declaration-clause \terminal{)} \opt{decl-specifier-seq}\br
    \bnfindent\opt{noexcept-specifier} \opt{attribute-specifier-seq} \opt{trailing-return-type}
\end{bnf}

\begin{bnf}
\nontermdef{lambda-capture}\br
    capture-default\br
    capture-list\br
    capture-default \terminal{,} capture-list
\end{bnf}

\begin{bnf}
\nontermdef{capture-default}\br
    \terminal{\&}\br
    \terminal{=}
\end{bnf}

\begin{bnf}
\nontermdef{capture-list}\br
    capture\br
    capture-list \terminal{,} capture
\end{bnf}

\begin{bnf}
\nontermdef{capture}\br
    simple-capture \opt{\terminal{...}}\br
    % FIXME: This is wrong. The ... should go after the &.
    \opt{\terminal{...}} init-capture
\end{bnf}

\begin{bnf}
\nontermdef{simple-capture}\br
    identifier\br
    \terminal{\&} identifier\br
    \terminal{this}\br
    \terminal{* this}
\end{bnf}

\begin{bnf}
\nontermdef{init-capture}\br
    identifier initializer\br
    \terminal{\&} identifier initializer
\end{bnf}

\begin{bnf}
\nontermdef{fold-expression}\br
    \terminal{(} cast-expression fold-operator \terminal{...} \terminal{)}\br
    \terminal{(} \terminal{...} fold-operator cast-expression \terminal{)}\br
    \terminal{(} cast-expression fold-operator \terminal{...} fold-operator cast-expression \terminal{)}
\end{bnf}

\begin{bnf}
%% Ed. note: character protrusion would misalign operators with leading `-`.
\microtypesetup{protrusion=false}
\nontermdef{fold-operator} \textnormal{one of}\br
    \terminal{+ }\quad\terminal{- }\quad\terminal{* }\quad\terminal{/ }\quad\terminal{\% }\quad\terminal{\caret{} }\quad\terminal{\& }\quad\terminal{| }\quad\terminal{<< }\quad\terminal{>> }\br
    \terminal{+=}\quad\terminal{-=}\quad\terminal{*=}\quad\terminal{/=}\quad\terminal{\%=}\quad\terminal{\caret=}\quad\terminal{\&=}\quad\terminal{|=}\quad\terminal{<<=}\quad\terminal{>>=}\quad\terminal{=}\br
    \terminal{==}\quad\terminal{!=}\quad\terminal{< }\quad\terminal{> }\quad\terminal{<=}\quad\terminal{>=}\quad\terminal{\&\&}\quad\terminal{||}\quad\terminal{,  }\quad\terminal{.* }\quad\terminal{->*}
\end{bnf}

\begin{bnf}
\nontermdef{requires-expression}\br
    \terminal{requires} \opt{requirement-parameter-list} requirement-body
\end{bnf}

\begin{bnf}
\nontermdef{requirement-parameter-list}\br
    \terminal{(} \opt{parameter-declaration-clause} \terminal{)}
\end{bnf}

\begin{bnf}
\nontermdef{requirement-body}\br
    \terminal{\{} requirement-seq \terminal{\}}
\end{bnf}

\begin{bnf}
\nontermdef{requirement-seq}\br
    requirement\br
    requirement-seq requirement
\end{bnf}

\begin{bnf}
\nontermdef{requirement}\br
    simple-requirement\br
    type-requirement\br
    compound-requirement\br
    nested-requirement
\end{bnf}

\begin{bnf}
\nontermdef{simple-requirement}\br
    expression \terminal{;}
\end{bnf}

\begin{bnf}
\nontermdef{type-requirement}\br
    \terminal{typename} \opt{nested-name-specifier} type-name \terminal{;}
\end{bnf}

\begin{bnf}
\nontermdef{compound-requirement}\br
    \terminal{\{} expression \terminal{\}} \opt{\terminal{noexcept}} \opt{return-type-requirement} \terminal{;}
\end{bnf}

\begin{bnf}
\nontermdef{return-type-requirement}\br
    trailing-return-type\br
    \terminal{->} \opt{cv-qualifier-seq} constrained-parameter \opt{cv-qualifier-seq} \opt{abstract-declarator}
\end{bnf}

\begin{bnf}
\nontermdef{nested-requirement}\br
    \terminal{requires} constraint-expression \terminal{;}
\end{bnf}

\begin{bnf}
\nontermdef{postfix-expression}\br
    primary-expression\br
    postfix-expression \terminal{[} expr-or-braced-init-list \terminal{]}\br
    postfix-expression \terminal{(} \opt{expression-list} \terminal{)}\br
    simple-type-specifier \terminal{(} \opt{expression-list} \terminal{)}\br
    typename-specifier \terminal{(} \opt{expression-list} \terminal{)}\br
    simple-type-specifier braced-init-list\br
    typename-specifier braced-init-list\br
    postfix-expression \opt{\terminal{. template}} id-expression\br
    postfix-expression \opt{\terminal{-> template}} id-expression\br
    postfix-expression \terminal{.} pseudo-destructor-name\br
    postfix-expression \terminal{->} pseudo-destructor-name\br
    postfix-expression \terminal{++}\br
    postfix-expression \terminal{-{-}}\br
    \terminal{dynamic_cast <} type-id \terminal{> (} expression \terminal{)}\br
    \terminal{static_cast <} type-id \terminal{> (} expression \terminal{)}\br
    \terminal{reinterpret_cast <} type-id \terminal{> (} expression \terminal{)}\br
    \terminal{const_cast <} type-id \terminal{> (} expression \terminal{)}\br
    atomicc-bit-cast \atomicc\br
    \terminal{typeid (} expression \terminal{)}\br
    \terminal{typeid (} type-id \terminal{)}
\end{bnf}

\begin{bnf}
\nontermdef{expression-list}\br
    initializer-list
\end{bnf}

\begin{bnf}
\nontermdef{pseudo-destructor-name}\br
    \opt{nested-name-specifier} type-name \terminal{::\,\~} type-name\br
    nested-name-specifier \terminal{template} simple-template-id \terminal{::\,\~} type-name\br
    \terminal{\~} type-name\br
    \terminal{\~} decltype-specifier
\end{bnf}

\begin{bnf}
\nontermdef{unary-expression}\br
    postfix-expression\br
    \terminal{++} cast-expression\br
    \terminal{-{-}} cast-expression\br
    unary-operator cast-expression\br
    \terminal{sizeof} unary-expression\br
    \terminal{sizeof (} type-id \terminal{)}\br
    \terminal{sizeof ...} \terminal{(} identifier \terminal{)}\br
    \terminal{alignof (} type-id \terminal{)}\br
    noexcept-expression\br
    new-expression\br
    delete-expression
\end{bnf}

\begin{bnf}
\nontermdef{unary-operator} \textnormal{one of}\br
    \terminal{*  \&  +  -  !  \~}
\end{bnf}

\begin{bnf}
\nontermdef{new-expression}\br
    \opt{\terminal{::}} \terminal{new} \opt{new-placement} new-type-id \opt{new-initializer} \br
    \opt{\terminal{::}} \terminal{new} \opt{new-placement} \terminal{(} type-id \terminal{)} \opt{new-initializer}
\end{bnf}

\begin{bnf}
\nontermdef{new-placement}\br
    \terminal{(} expression-list \terminal{)}
\end{bnf}

\begin{bnf}
\nontermdef{new-type-id}\br
    type-specifier-seq \opt{new-declarator}
\end{bnf}

\begin{bnf}
\nontermdef{new-declarator}\br
    ptr-operator \opt{new-declarator} \br
    noptr-new-declarator
\end{bnf}

\begin{bnf}
\nontermdef{noptr-new-declarator}\br
    \terminal{[} expression \terminal{]} \opt{attribute-specifier-seq}\br
    noptr-new-declarator \terminal{[} constant-expression \terminal{]} \opt{attribute-specifier-seq}
\end{bnf}

\begin{bnf}
\nontermdef{new-initializer}\br
    \terminal{(} \opt{expression-list} \terminal{)}\br
    braced-init-list
\end{bnf}

\begin{bnf}
\nontermdef{delete-expression}\br
    \opt{\terminal{::}} \terminal{delete} cast-expression\br
    \opt{\terminal{::}} \terminal{delete [ ]} cast-expression
\end{bnf}

\begin{bnf}
\nontermdef{noexcept-expression}\br
  \terminal{noexcept} \terminal{(} expression \terminal{)}
\end{bnf}

\begin{bnf}
\nontermdef{cast-expression}\br
    unary-expression\br
    \terminal{(} type-id \terminal{)} cast-expression
\end{bnf}

\begin{bnf}
\nontermdef{pm-expression}\br
    cast-expression\br
    pm-expression \terminal{.*} cast-expression\br
    pm-expression \terminal{->*} cast-expression
\end{bnf}

\begin{bnf}
\nontermdef{multiplicative-expression}\br
    pm-expression\br
    multiplicative-expression \terminal{*} pm-expression\br
    multiplicative-expression \terminal{/} pm-expression\br
    multiplicative-expression \terminal{\%} pm-expression
\end{bnf}

\begin{bnf}
\nontermdef{additive-expression}\br
    multiplicative-expression\br
    additive-expression \terminal{+} multiplicative-expression\br
    additive-expression \terminal{-} multiplicative-expression
\end{bnf}

\begin{bnf}
\nontermdef{shift-expression}\br
    additive-expression\br
    shift-expression \terminal{<<} additive-expression\br
    shift-expression \terminal{>>} additive-expression
\end{bnf}

\begin{bnf}
\nontermdef{compare-expression}\br
    shift-expression\br
    compare-expression \terminal{<=>} shift-expression
\end{bnf}

\begin{bnf}
\nontermdef{relational-expression}\br
    compare-expression\br
    relational-expression \terminal{<} compare-expression\br
    relational-expression \terminal{>} compare-expression\br
    relational-expression \terminal{<=} compare-expression\br
    relational-expression \terminal{>=} compare-expression
\end{bnf}

\begin{bnf}
\nontermdef{equality-expression}\br
    relational-expression\br
    equality-expression \terminal{==} relational-expression\br
    equality-expression \terminal{!=} relational-expression
\end{bnf}

\begin{bnf}
\nontermdef{and-expression}\br
    equality-expression\br
    and-expression \terminal{\&} equality-expression
\end{bnf}

\begin{bnf}
\nontermdef{exclusive-or-expression}\br
    and-expression\br
    exclusive-or-expression \terminal{\caret} and-expression
\end{bnf}

\begin{bnf}
\nontermdef{inclusive-or-expression}\br
    exclusive-or-expression\br
    inclusive-or-expression \terminal{|} exclusive-or-expression
\end{bnf}

\begin{bnf}
\nontermdef{logical-and-expression}\br
    inclusive-or-expression\br
    logical-and-expression \terminal{\&\&} inclusive-or-expression
\end{bnf}

\begin{bnf}
\nontermdef{logical-or-expression}\br
    logical-and-expression\br
    logical-or-expression \terminal{||} logical-and-expression
\end{bnf}

\begin{bnf}
\nontermdef{conditional-expression}\br
    logical-or-expression\br
    logical-or-expression \terminal{?} expression \terminal{:} assignment-expression
\end{bnf}

\begin{bnf}
\nontermdef{throw-expression}\br
    \terminal{throw}  \opt{assignment-expression}
\end{bnf}

\begin{bnf}
\nontermdef{assignment-expression}\br
    conditional-expression\br
    logical-or-expression assignment-operator initializer-clause\br
    throw-expression
\end{bnf}

\begin{bnf}
\nontermdef{assignment-operator} \textnormal{one of}\br
    \terminal{=  *=  /=  \%=   +=  -=  >>=  <<=  \&=  \caret=  |=}
\end{bnf}

\begin{bnf}
\nontermdef{expression}\br
    assignment-expression\br
    expression \terminal{,} assignment-expression
\end{bnf}

\begin{bnf}
\nontermdef{constant-expression}\br
    conditional-expression
\end{bnf}

\gramSec[gram.stmt]{Statements}

\begin{bnf}
\nontermdef{statement}\br
    labeled-statement\br
    \opt{attribute-specifier-seq} expression-statement\br
    \opt{attribute-specifier-seq} compound-statement\br
    \opt{attribute-specifier-seq} selection-statement\br
    \opt{attribute-specifier-seq} iteration-statement\br
    \opt{attribute-specifier-seq} jump-statement\br
    declaration-statement\br
    \opt{attribute-specifier-seq} try-block\br
    rule-statement \atomicc\br

\nontermdef{init-statement}\br
    expression-statement\br
    simple-declaration

\nontermdef{condition}\br
    expression\br
    \opt{attribute-specifier-seq} decl-specifier-seq declarator brace-or-equal-initializer
\end{bnf}

\begin{bnf}
\nontermdef{labeled-statement}\br
    \opt{attribute-specifier-seq} identifier \terminal{:} statement\br
    \opt{attribute-specifier-seq} \terminal{case} constant-expression \terminal{:} statement\br
    \opt{attribute-specifier-seq} \terminal{default :} statement
\end{bnf}

\begin{bnf}
\nontermdef{expression-statement}\br
    \opt{expression} \terminal{;}
\end{bnf}

\begin{bnf}
\nontermdef{compound-statement}\br
    \terminal{\{} \opt{statement-seq} \terminal{\}}
\end{bnf}

\begin{bnf}
\nontermdef{statement-seq}\br
    statement\br
    statement-seq statement
\end{bnf}

\begin{bnf}
\nontermdef{selection-statement}\br
    \terminal{if \opt{constexpr} (} \opt{init-statement} condition \terminal{)} statement\br
    \terminal{if \opt{constexpr} (} \opt{init-statement} condition \terminal{)} statement \terminal{else} statement\br
    \terminal{switch (} \opt{init-statement} condition \terminal{)} statement
\end{bnf}

\begin{bnf}
\nontermdef{iteration-statement}\br
    \terminal{while (} condition \terminal{)} statement\br
    \terminal{do} statement \terminal{while (} expression \terminal{) ;}\br
    \terminal{for (} init-statement \opt{condition} \terminal{;} \opt{expression} \terminal{)} statement\br
    \terminal{for (} \opt{init-statement} for-range-declaration \terminal{:} for-range-initializer \terminal{)} statement
\end{bnf}

\begin{bnf}
\nontermdef{for-range-declaration}\br
    \opt{attribute-specifier-seq} decl-specifier-seq declarator\br
    \opt{attribute-specifier-seq} decl-specifier-seq \opt{ref-qualifier} \terminal{[} identifier-list \terminal{]}
\end{bnf}

\begin{bnf}
\nontermdef{for-range-initializer}\br
    expr-or-braced-init-list
\end{bnf}

\begin{bnf}
\nontermdef{jump-statement}\br
    \terminal{break ;}\br
    \terminal{continue ;}\br
    \terminal{return} \opt{expr-or-braced-init-list} \terminal{;}\br
    \terminal{goto} identifier \terminal{;}
\end{bnf}

\begin{bnf}
\nontermdef{declaration-statement}\br
    block-declaration
\end{bnf}

\gramSec[gram.dcl]{Declarations}

\begin{bnf}
\nontermdef{declaration-seq}\br
    declaration\br
    declaration-seq declaration
\end{bnf}

\begin{bnf}
\nontermdef{declaration}\br
    block-declaration\br
    nodeclspec-function-declaration\br
    function-definition\br
    template-declaration\br
    deduction-guide\br
    explicit-instantiation\br
    explicit-specialization\br
    linkage-specification\br
    namespace-definition\br
    empty-declaration\br
    attribute-declaration
\end{bnf}

\begin{bnf}
\nontermdef{block-declaration}\br
    simple-declaration\br
    asm-definition\br
    namespace-alias-definition\br
    using-declaration\br
    using-directive\br
    static_assert-declaration\br
    alias-declaration\br
    opaque-enum-declaration
\end{bnf}

\begin{bnf}
\nontermdef{nodeclspec-function-declaration}\br
    \opt{attribute-specifier-seq} declarator \terminal{;}
\end{bnf}

\begin{bnf}
\nontermdef{alias-declaration}\br
    \terminal{using} identifier \opt{attribute-specifier-seq} \terminal{=} defining-type-id \terminal{;}
\end{bnf}

\begin{bnf}
\nontermdef{simple-declaration}\br
    decl-specifier-seq \opt{init-declarator-list} \terminal{;}\br
    attribute-specifier-seq decl-specifier-seq init-declarator-list \terminal{;}\br
    \opt{attribute-specifier-seq} decl-specifier-seq \opt{ref-qualifier} \terminal{[} identifier-list \terminal{]} initializer \terminal{;}
\end{bnf}

\begin{bnf}
\nontermdef{static_assert-declaration}\br
  \terminal{static_assert} \terminal{(} constant-expression \terminal{)} \terminal{;}\br
  \terminal{static_assert} \terminal{(} constant-expression \terminal{,} string-literal \terminal{)} \terminal{;}
\end{bnf}

\begin{bnf}
\nontermdef{empty-declaration}\br
    \terminal{;}
\end{bnf}

\begin{bnf}
\nontermdef{attribute-declaration}\br
    attribute-specifier-seq \terminal{;}
\end{bnf}

\begin{bnf}
\nontermdef{decl-specifier}\br
    storage-class-specifier\br
    defining-type-specifier\br
    function-specifier\br
    \terminal{friend}\br
    \terminal{typedef}\br
    \terminal{constexpr}\br
    \terminal{inline}
\end{bnf}

\begin{bnf}
\nontermdef{decl-specifier-seq}\br
    decl-specifier \opt{attribute-specifier-seq}\br
    decl-specifier decl-specifier-seq
\end{bnf}

\begin{bnf}
\nontermdef{storage-class-specifier}\br
    \terminal{static}\br
    \terminal{thread_local}\br
    \terminal{extern}\br
    \terminal{mutable}
\end{bnf}

\begin{bnf}
\nontermdef{function-specifier}\br
    \terminal{virtual}\br
    explicit-specifier
\end{bnf}

\begin{bnf}
\nontermdef{explicit-specifier}\br
    \terminal{explicit} \terminal{(} constant-expression \terminal{)}\br
    \terminal{explicit}
\end{bnf}

\begin{bnf}
\nontermdef{typedef-name}\br
    identifier
\end{bnf}

\begin{bnf}
\nontermdef{type-specifier}\br
  simple-type-specifier\br
  elaborated-type-specifier\br
  typename-specifier\br
  cv-qualifier
\end{bnf}

\begin{bnf}
\nontermdef{type-specifier-seq}\br
    type-specifier \opt{attribute-specifier-seq}\br
    type-specifier type-specifier-seq
\end{bnf}

\begin{bnf}
\nontermdef{defining-type-specifier}\br
    type-specifier\br
    class-specifier\br
    enum-specifier
\end{bnf}

\begin{bnf}
\nontermdef{defining-type-specifier-seq}\br
  defining-type-specifier \opt{attribute-specifier-seq}\br
  defining-type-specifier defining-type-specifier-seq
\end{bnf}

\begin{bnf}
\nontermdef{simple-type-specifier}\br
    \opt{nested-name-specifier} type-name\br
    nested-name-specifier \terminal{template} simple-template-id\br
    \opt{nested-name-specifier} template-name\br
    \terminal{char}\br
    \terminal{char16_t}\br
    \terminal{char32_t}\br
    \terminal{wchar_t}\br
    \terminal{bool}\br
    \terminal{short}\br
    \terminal{int}\br
    \terminal{long}\br
    \terminal{signed}\br
    \terminal{unsigned}\br
    \terminal{float}\br
    \terminal{double}\br
    \terminal{void}\br
    \terminal{auto}\br
    decltype-specifier\br
    bit-type-specifier \atomicc
\end{bnf}

\begin{bnf}
\nontermdef{type-name}\br
    class-name\br
    enum-name\br
    typedef-name\br
    simple-template-id
\end{bnf}

\begin{bnf}
\nontermdef{decltype-specifier}\br
  \terminal{decltype} \terminal{(} expression \terminal{)}\br
  \terminal{decltype} \terminal{(} \terminal{auto} \terminal{)}
\end{bnf}

\begin{bnf}
\nontermdef{elaborated-type-specifier}\br
    class-key \opt{attribute-specifier-seq} \opt{nested-name-specifier} identifier\br
    class-key simple-template-id\br
    class-key nested-name-specifier \opt{\terminal{template}} simple-template-id\br
    \terminal{enum} \opt{nested-name-specifier} identifier
\end{bnf}

\begin{bnf}
\nontermdef{init-declarator-list}\br
    init-declarator\br
    init-declarator-list \terminal{,} init-declarator
\end{bnf}

\begin{bnf}
\nontermdef{init-declarator}\br
    declarator \opt{initializer}\br
    declarator requires-clause
\end{bnf}

\begin{bnf}
\nontermdef{declarator}\br
    ptr-declarator\br
    noptr-declarator parameters-and-qualifiers trailing-return-type
\end{bnf}

\begin{bnf}
\nontermdef{ptr-declarator}\br
    noptr-declarator\br
    ptr-operator ptr-declarator
\end{bnf}

\begin{bnf}
\nontermdef{noptr-declarator}\br
    declarator-id \opt{attribute-specifier-seq}\br
    noptr-declarator parameters-and-qualifiers\br
    noptr-declarator \terminal{[} \opt{constant-expression} \terminal{]} \opt{attribute-specifier-seq}\br
    \terminal{(} ptr-declarator \terminal{)}
\end{bnf}

\begin{bnf}
\nontermdef{parameters-and-qualifiers}\br
    \terminal{(} parameter-declaration-clause \terminal{)} \opt{cv-qualifier-seq}\br
    \bnfindent\opt{ref-qualifier} \opt{noexcept-specifier} \opt{attribute-specifier-seq}
\end{bnf}

\begin{bnf}
\nontermdef{trailing-return-type}\br
    \terminal{->} type-id
\end{bnf}

\begin{bnf}
\nontermdef{ptr-operator}\br
    \terminal{*} \opt{attribute-specifier-seq} \opt{cv-qualifier-seq}\br
    \terminal{\&} \opt{attribute-specifier-seq}\br
    \terminal{\&\&} \opt{attribute-specifier-seq}\br
    nested-name-specifier \terminal{*} \opt{attribute-specifier-seq} \opt{cv-qualifier-seq}
\end{bnf}

\begin{bnf}
\nontermdef{cv-qualifier-seq}\br
    cv-qualifier \opt{cv-qualifier-seq}
\end{bnf}

\begin{bnf}
\nontermdef{cv-qualifier}\br
    \terminal{const}\br
    \terminal{volatile}
\end{bnf}

\begin{bnf}
\nontermdef{ref-qualifier}\br
    \terminal{\&}\br
    \terminal{\&\&}
\end{bnf}

\begin{bnf}
\nontermdef{declarator-id}\br
    \opt{\terminal{...}} id-expression
\end{bnf}

\begin{bnf}
\nontermdef{type-id}\br
    type-specifier-seq \opt{abstract-declarator}
\end{bnf}

\begin{bnf}
\nontermdef{defining-type-id}\br
    defining-type-specifier-seq \opt{abstract-declarator}
\end{bnf}

\begin{bnf}
\nontermdef{abstract-declarator}\br
    ptr-abstract-declarator\br
    \opt{noptr-abstract-declarator} parameters-and-qualifiers trailing-return-type\br
    abstract-pack-declarator
\end{bnf}

\begin{bnf}
\nontermdef{ptr-abstract-declarator}\br
    noptr-abstract-declarator\br
    ptr-operator \opt{ptr-abstract-declarator}
\end{bnf}

\begin{bnf}
\nontermdef{noptr-abstract-declarator}\br
    \opt{noptr-abstract-declarator} parameters-and-qualifiers\br
    \opt{noptr-abstract-declarator} \terminal{[} \opt{constant-expression} \terminal{]} \opt{attribute-specifier-seq}\br
    \terminal{(} ptr-abstract-declarator \terminal{)}
\end{bnf}

\begin{bnf}
\nontermdef{abstract-pack-declarator}\br
    noptr-abstract-pack-declarator\br
    ptr-operator abstract-pack-declarator
\end{bnf}

\begin{bnf}
\nontermdef{noptr-abstract-pack-declarator}\br
    noptr-abstract-pack-declarator parameters-and-qualifiers\br
    noptr-abstract-pack-declarator \terminal{[} \opt{constant-expression} \terminal{]} \opt{attribute-specifier-seq}\br
    \terminal{...}
\end{bnf}

\begin{bnf}
\nontermdef{parameter-declaration-clause}\br
    \opt{parameter-declaration-list} \opt{\terminal{...}}\br
    parameter-declaration-list \terminal{, ...}
\end{bnf}

\begin{bnf}
\nontermdef{parameter-declaration-list}\br
    parameter-declaration\br
    parameter-declaration-list \terminal{,} parameter-declaration
\end{bnf}

\begin{bnf}
\nontermdef{parameter-declaration}\br
    \opt{attribute-specifier-seq} decl-specifier-seq declarator\br
    \opt{attribute-specifier-seq} decl-specifier-seq declarator \terminal{=} initializer-clause\br
    \opt{attribute-specifier-seq} decl-specifier-seq \opt{abstract-declarator}\br
    \opt{attribute-specifier-seq} decl-specifier-seq \opt{abstract-declarator} \terminal{=} initializer-clause
\end{bnf}

\begin{bnf}
\nontermdef{initializer}\br
    brace-or-equal-initializer\br
    \terminal{(} expression-list \terminal{)}
\end{bnf}

\begin{bnf}
\nontermdef{brace-or-equal-initializer}\br
    \terminal{=} initializer-clause\br
    braced-init-list
\end{bnf}

\begin{bnf}
\nontermdef{initializer-clause}\br
    assignment-expression\br
    braced-init-list
\end{bnf}

\begin{bnf}
\nontermdef{braced-init-list}\br
    \terminal{\{} initializer-list \opt{\terminal{,}} \terminal{\}}\br
    \terminal{\{} designated-initializer-list \opt{\terminal{,}} \terminal{\}}\br
    \terminal{\{} \terminal{\}}
\end{bnf}

\begin{bnf}
\nontermdef{initializer-list}\br
    initializer-clause \opt{\terminal{...}}\br
    initializer-list \terminal{,} initializer-clause \opt{\terminal{...}}
\end{bnf}

\begin{bnf}
\nontermdef{designated-initializer-list}\br
    designated-initializer-clause\br
    designated-initializer-list \terminal{,} designated-initializer-clause
\end{bnf}

\begin{bnf}
\nontermdef{designated-initializer-clause}\br
    designator brace-or-equal-initializer
\end{bnf}

\begin{bnf}
\nontermdef{designator}\br
    \terminal{.} identifier
\end{bnf}

\begin{bnf}
\nontermdef{expr-or-braced-init-list}\br
    expression\br
    braced-init-list
\end{bnf}

\begin{bnf}
\nontermdef{function-definition}\br
    atomicc-method-definition \atomicc\br
    \opt{attribute-specifier-seq} \opt{decl-specifier-seq} declarator \opt{virt-specifier-seq} function-body\br
    \opt{attribute-specifier-seq} \opt{decl-specifier-seq} declarator requires-clause function-body
\end{bnf}

\begin{bnf}
\nontermdef{function-body}\br
    atomicc-function-body \atomicc\br
    function-try-block\br
    \terminal{= default ;}\br
    \terminal{= delete ;}
\end{bnf}

\begin{bnf}
\nontermdef{enum-name}\br
    identifier
\end{bnf}

\begin{bnf}
\nontermdef{enum-specifier}\br
    enum-head \terminal{\{} \opt{enumerator-list} \terminal{\}}\br
    enum-head \terminal{\{} enumerator-list \terminal{, \}}
\end{bnf}

\begin{bnf}
\nontermdef{enum-head}\br
    enum-key \opt{attribute-specifier-seq} \opt{enum-head-name} \opt{enum-base}
\end{bnf}

\begin{bnf}
\nontermdef{enum-head-name}\br
    \opt{nested-name-specifier} identifier
\end{bnf}

\begin{bnf}
\nontermdef{opaque-enum-declaration}\br
    enum-key \opt{attribute-specifier-seq} \opt{nested-name-specifier} identifier \opt{enum-base} \terminal{;}
\end{bnf}

\begin{bnf}
\nontermdef{enum-key}\br
    \terminal{enum}\br
    \terminal{enum class}\br
    \terminal{enum struct}
\end{bnf}

\begin{bnf}
\nontermdef{enum-base}\br
    \terminal{:} type-specifier-seq
\end{bnf}

\begin{bnf}
\nontermdef{enumerator-list}\br
    enumerator-definition\br
    enumerator-list \terminal{,} enumerator-definition
\end{bnf}

\begin{bnf}
\nontermdef{enumerator-definition}\br
    enumerator\br
    enumerator \terminal{=} constant-expression
\end{bnf}

\begin{bnf}
\nontermdef{enumerator}\br
    identifier \opt{attribute-specifier-seq}
\end{bnf}

\begin{bnf}
\nontermdef{namespace-name}\br
        identifier\br
        namespace-alias
\end{bnf}

\begin{bnf}
\nontermdef{namespace-definition}\br
        named-namespace-definition\br
        unnamed-namespace-definition\br
        nested-namespace-definition
\end{bnf}

\begin{bnf}
\nontermdef{named-namespace-definition}\br
        \terminal{\opt{inline}} \terminal{namespace} \opt{attribute-specifier-seq} identifier \terminal{\{} namespace-body \terminal{\}}
\end{bnf}

\begin{bnf}
\nontermdef{unnamed-namespace-definition}\br
        \terminal{\opt{inline}} \terminal{namespace} \opt{attribute-specifier-seq} \terminal{\{} namespace-body \terminal{\}}
\end{bnf}

\begin{bnf}
\nontermdef{nested-namespace-definition}\br
        \terminal{namespace} enclosing-namespace-specifier \terminal{::} identifier \terminal{\{} namespace-body \terminal{\}}
\end{bnf}

\begin{bnf}
\nontermdef{enclosing-namespace-specifier}\br
        identifier\br
        enclosing-namespace-specifier \terminal{::} identifier
\end{bnf}

\begin{bnf}
\nontermdef{namespace-body}\br
        \opt{declaration-seq}
\end{bnf}

\begin{bnf}
\nontermdef{namespace-alias}\br
        identifier
\end{bnf}

\begin{bnf}
\nontermdef{namespace-alias-definition}\br
        \terminal{namespace} identifier \terminal{=} qualified-namespace-specifier \terminal{;}
\end{bnf}

\begin{bnf}
\nontermdef{qualified-namespace-specifier}\br
    \opt{nested-name-specifier} namespace-name
\end{bnf}

\begin{bnf}
\nontermdef{using-directive}\br
    \opt{attribute-specifier-seq} \terminal{using  namespace} \opt{nested-name-specifier} namespace-name \terminal{;}
\end{bnf}

\begin{bnf}
\nontermdef{using-declaration}\br
    \terminal{using} using-declarator-list \terminal{;}
\end{bnf}

\begin{bnf}
\nontermdef{using-declarator-list}\br
    using-declarator \opt{\terminal{...}}\br
    using-declarator-list \terminal{,} using-declarator \opt{\terminal{...}}
\end{bnf}

\begin{bnf}
\nontermdef{using-declarator}\br
    \terminal{\opt{typename}} nested-name-specifier unqualified-id
\end{bnf}

\begin{bnf}
\nontermdef{asm-definition}\br
    \opt{attribute-specifier-seq} \terminal{asm (} string-literal \terminal{) ;}
\end{bnf}

\begin{bnf}
\nontermdef{linkage-specification}\br
    \terminal{extern} string-literal \terminal{\{} \opt{declaration-seq} \terminal{\}}\br
    \terminal{extern} string-literal declaration
\end{bnf}

\begin{bnf}
\nontermdef{attribute-specifier-seq}\br
  \opt{attribute-specifier-seq} attribute-specifier
\end{bnf}

\begin{bnf}
\nontermdef{attribute-specifier}\br
  \terminal{[} \terminal{[} \opt{attribute-using-prefix} attribute-list \terminal{]} \terminal{]}\br
  contract-attribute-specifier\br
  alignment-specifier
\end{bnf}

\begin{bnf}
\nontermdef{alignment-specifier}\br
  \terminal{alignas (} type-id \opt{\terminal{...}} \terminal{)}\br
  \terminal{alignas (} constant-expression \opt{\terminal{...}} \terminal{)}
\end{bnf}

\begin{bnf}
\nontermdef{attribute-using-prefix}\br
  \terminal{using} attribute-namespace \terminal{:}
\end{bnf}

\begin{bnf}
\nontermdef{attribute-list}\br
  \opt{attribute}\br
  attribute-list \terminal{,} \opt{attribute}\br
  attribute \terminal{...}\br
  attribute-list \terminal{,} attribute \terminal{...}
\end{bnf}

\begin{bnf}
\nontermdef{attribute}\br
    attribute-token \opt{attribute-argument-clause}
\end{bnf}

\begin{bnf}
\nontermdef{attribute-token}\br
    identifier\br
    attribute-scoped-token
\end{bnf}

\begin{bnf}
\nontermdef{attribute-scoped-token}\br
    attribute-namespace \terminal{::} identifier
\end{bnf}

\begin{bnf}
\nontermdef{attribute-namespace}\br
    identifier
\end{bnf}

\begin{bnf}
\nontermdef{attribute-argument-clause}\br
    \terminal{(} \opt{balanced-token-seq} \terminal{)}
\end{bnf}

\begin{bnf}
\nontermdef{balanced-token-seq}\br
    balanced-token\br
    balanced-token-seq balanced-token
\end{bnf}

\begin{bnf}
\nontermdef{balanced-token}\br
    \terminal{(} \opt{balanced-token-seq} \terminal{)}\br
    \terminal{[} \opt{balanced-token-seq} \terminal{]}\br
    \terminal{\{} \opt{balanced-token-seq} \terminal{\}}\br
    \textnormal{any \grammarterm{token} other than a parenthesis, a bracket, or a brace}
\end{bnf}

\begin{bnf}
\nontermdef{contract-attribute-specifier}\br
  \terminal{[} \terminal{[} \terminal{expects} \opt{contract-level} \terminal{:} conditional-expression \terminal{]} \terminal{]}\br
  \terminal{[} \terminal{[} \terminal{ensures} \opt{contract-level} \opt{identifier} \terminal{:} conditional-expression \terminal{]} \terminal{]}\br
    \terminal{[} \terminal{[} \terminal{assert} \opt{contract-level} \terminal{:} conditional-expression \terminal{]} \terminal{]}
\end{bnf}

\begin{bnf}
\nontermdef{contract-level}\br
  \terminal{default}\br
  \terminal{audit}\br
  \terminal{axiom}
\end{bnf}

\gramSec[gram.class]{Classes}

\begin{bnf}
\nontermdef{class-name}\br
    identifier\br
    simple-template-id
\end{bnf}

\begin{bnf}
\nontermdef{class-specifier}\br
    class-head \terminal{\{} \opt{member-specification} \terminal{\}}
\end{bnf}

\begin{bnf}
\nontermdef{class-head}\br
    class-key \opt{attribute-specifier-seq} class-head-name \opt{class-virt-specifier} \opt{base-clause}\br
    class-key \opt{attribute-specifier-seq} \opt{base-clause}
\end{bnf}

\begin{bnf}
\nontermdef{class-head-name}\br
    \opt{nested-name-specifier} class-name
\end{bnf}

\begin{bnf}
\nontermdef{class-virt-specifier}\br
    \terminal{final}
\end{bnf}

\begin{bnf}
\nontermdef{class-key}\br
    \terminal{class}\br
    \terminal{struct}\br
    \terminal{union}\br
    atomicc-class-key \atomicc
\end{bnf}

\begin{bnf}
\nontermdef{member-specification}\br
    member-declaration \opt{member-specification}\br
    access-specifier \terminal{:} \opt{member-specification}
\end{bnf}

\begin{bnf}
\nontermdef{member-declaration}\br
    \opt{attribute-specifier-seq} \opt{decl-specifier-seq} \opt{member-declarator-list} \terminal{;}\br
    atomicc-method-declaration \atomicc\br
    function-definition\br
    using-declaration\br
    static_assert-declaration\br
    template-declaration\br
    deduction-guide\br
    alias-declaration\br
    connect-declaration \atomicc \br
    printf-declaration \atomicc \br
    empty-declaration
\end{bnf}

\begin{bnf}
\nontermdef{member-declarator-list}\br
    member-declarator\br
    member-declarator-list \terminal{,} member-declarator
\end{bnf}

\begin{bnf}
\nontermdef{member-declarator}\br
    declarator \opt{virt-specifier-seq} \opt{pure-specifier}\br
    declarator requires-clause\br
    declarator \opt{brace-or-equal-initializer}\br
    \opt{identifier} \opt{attribute-specifier-seq} \terminal{:} constant-expression \opt{brace-or-equal-initializer}
\end{bnf}

\begin{bnf}
\nontermdef{virt-specifier-seq}\br
    virt-specifier\br
    virt-specifier-seq virt-specifier
\end{bnf}

\begin{bnf}
\nontermdef{virt-specifier}\br
    \terminal{override}\br
    \terminal{final}
\end{bnf}

\begin{bnf}
\nontermdef{pure-specifier}\br
    \terminal{= 0}
\end{bnf}

\begin{bnf}
\nontermdef{conversion-function-id}\br
    \terminal{operator} conversion-type-id
\end{bnf}

\begin{bnf}
\nontermdef{conversion-type-id}\br
    type-specifier-seq \opt{conversion-declarator}
\end{bnf}

\begin{bnf}
\nontermdef{conversion-declarator}\br
    ptr-operator \opt{conversion-declarator}
\end{bnf}

\begin{bnf}
\nontermdef{base-clause}\br
    \terminal{:} base-specifier-list
\end{bnf}

\begin{bnf}
\nontermdef{base-specifier-list}\br
    base-specifier \opt{\terminal{...}}\br
    base-specifier-list \terminal{,} base-specifier \opt{\terminal{...}}
\end{bnf}

\begin{bnf}
\nontermdef{base-specifier}\br
    \opt{attribute-specifier-seq} class-or-decltype\br
    \opt{attribute-specifier-seq} \terminal{virtual} \opt{access-specifier} class-or-decltype\br
    \opt{attribute-specifier-seq} access-specifier \opt{\terminal{virtual}} class-or-decltype
\end{bnf}

\begin{bnf}
\nontermdef{class-or-decltype}\br
    \opt{nested-name-specifier} class-name\br
    nested-name-specifier \terminal{template} simple-template-id\br
    decltype-specifier
\end{bnf}

\begin{bnf}
\nontermdef{access-specifier}\br
    \terminal{private}\br
    \terminal{protected}\br
    \terminal{public}
\end{bnf}

\begin{bnf}
\nontermdef{ctor-initializer}\br
    \terminal{:} mem-initializer-list
\end{bnf}

\begin{bnf}
\nontermdef{mem-initializer-list}\br
    mem-initializer \opt{\terminal{...}}\br
    mem-initializer-list \terminal{,} mem-initializer \opt{\terminal{...}}
\end{bnf}

\begin{bnf}
\nontermdef{mem-initializer}\br
    mem-initializer-id \terminal{(} \opt{expression-list} \terminal{)}\br
    mem-initializer-id braced-init-list
\end{bnf}

\begin{bnf}
\nontermdef{mem-initializer-id}\br
    class-or-decltype\br
    identifier
\end{bnf}

\gramSec[gram.over]{Overloading}

\begin{bnf}
\nontermdef{operator-function-id}\br
    \terminal{operator} operator
\end{bnf}

\begin{bnf}
%% Ed. note: character protrusion would misalign various operators.
\microtypesetup{protrusion=false}\obeyspaces
\nontermdef{operator} \textnormal{one of}\br
    \terminal{new      delete   new[]    delete[] (\rlap{\,)}        [\rlap{\,]}        ->       ->*      \~}\br
    \terminal{!        +        -        *        /        \%        \caret{}        \&        |}\br
    \terminal{=        +=       -=       *=       /=       \%=       \caret{}=       \&=       |=}\br
    \terminal{==       !=       <        >        <=       >=       <=>      \&\&       ||}\br
    \terminal{<<       >>       <<=      >>=      ++       --       ,}\br
\end{bnf}

\begin{bnf}
\nontermdef{literal-operator-id}\br
    \terminal{operator} string-literal identifier\br
    \terminal{operator} user-defined-string-literal
\end{bnf}

\gramSec[gram.temp]{Templates}

\begin{bnf}
\nontermdef{template-declaration}\br
  template-head declaration\br
  template-head concept-definition
\end{bnf}

\begin{bnf}
\nontermdef{template-head}\br
  \terminal{template} \terminal{<} template-parameter-list \terminal{>} \opt{requires-clause}
\end{bnf}

\begin{bnf}
\nontermdef{template-parameter-list}\br
  template-parameter\br
  template-parameter-list \terminal{,} template-parameter
\end{bnf}

\begin{bnf}
\nontermdef{requires-clause}\br
  \terminal{requires} constraint-logical-or-expression
\end{bnf}

\begin{bnf}
\nontermdef{constraint-logical-or-expression}\br
  constraint-logical-and-expression\br
  constraint-logical-or-expression \terminal{||} constraint-logical-and-expression
\end{bnf}

\begin{bnf}
\nontermdef{constraint-logical-and-expression}\br
  primary-expression\br
  constraint-logical-and-expression \terminal{\&\&} primary-expression
\end{bnf}

\begin{bnf}
\nontermdef{concept-definition}\br
  \terminal{concept} concept-name \terminal{=} constraint-expression \terminal{;}
\end{bnf}

\begin{bnf}
\nontermdef{concept-name}\br
  identifier
\end{bnf}

\begin{bnf}
\nontermdef{template-parameter}\br
  type-parameter\br
  parameter-declaration\br
  constrained-parameter
\end{bnf}

\begin{bnf}
\nontermdef{type-parameter}\br
  type-parameter-key \opt{\terminal{...}} \opt{identifier}\br
  type-parameter-key \opt{identifier} \terminal{=} type-id\br
  template-head type-parameter-key \opt{\terminal{...}} \opt{identifier}\br
  template-head type-parameter-key \opt{identifier} \terminal{=} id-expression
\end{bnf}

\begin{bnf}
\nontermdef{type-parameter-key}\br
  \terminal{class}\br
  \terminal{typename}
\end{bnf}

\begin{bnf}
\nontermdef{constrained-parameter}\br
  qualified-concept-name \terminal{...} \opt{identifier}\br
  qualified-concept-name \opt{identifier} \opt{default-template-argument}
\end{bnf}

\begin{bnf}
\nontermdef{qualified-concept-name}\br
  \opt{nested-name-specifier} concept-name\br
  \opt{nested-name-specifier} partial-concept-id
\end{bnf}

\begin{bnf}
\nontermdef{partial-concept-id}\br
  concept-name \terminal{<} \opt{template-argument-list} \terminal{>}
\end{bnf}

\begin{bnf}
\nontermdef{default-template-argument}\br
  \terminal{=} type-id\br
  \terminal{=} id-expression\br
  \terminal{=} initializer-clause
\end{bnf}

\begin{bnf}
\nontermdef{simple-template-id}\br
  template-name \terminal{<} \opt{template-argument-list} \terminal{>}
\end{bnf}

\begin{bnf}
\nontermdef{template-id}\br
  simple-template-id\br
  operator-function-id \terminal{<} \opt{template-argument-list} \terminal{>}\br
  literal-operator-id \terminal{<} \opt{template-argument-list} \terminal{>}
\end{bnf}

\begin{bnf}
\nontermdef{template-name}\br
  identifier
\end{bnf}

\begin{bnf}
\nontermdef{template-argument-list}\br
  template-argument \opt{\terminal{...}}\br
  template-argument-list \terminal{,} template-argument \opt{\terminal{...}}
\end{bnf}

\begin{bnf}
\nontermdef{template-argument}\br
  constant-expression\br
  type-id\br
  id-expression
\end{bnf}

\begin{bnf}
\nontermdef{constraint-expression}\br
    logical-or-expression
\end{bnf}

\begin{bnf}
\nontermdef{typename-specifier}\br
  \terminal{typename} nested-name-specifier identifier\br
  \terminal{typename} nested-name-specifier \terminal{\opt{template}} simple-template-id
\end{bnf}

\begin{bnf}
\nontermdef{explicit-instantiation}\br
  \terminal{\opt{extern}} \terminal{template} declaration
\end{bnf}

\begin{bnf}
\nontermdef{explicit-specialization}\br
  \terminal{template < >} declaration
\end{bnf}

\begin{bnf}
\nontermdef{deduction-guide}\br
    \opt{\terminal{explicit}} template-name \terminal{(} parameter-declaration-clause \terminal{) ->} simple-template-id \terminal{;}
\end{bnf}

\gramSec[gram.except]{Exception handling}

\begin{bnf}
\nontermdef{try-block}\br
    \terminal{try} compound-statement handler-seq
\end{bnf}

\begin{bnf}
\nontermdef{function-try-block}\br
    \terminal{try} \opt{ctor-initializer} compound-statement handler-seq
\end{bnf}

\begin{bnf}
\nontermdef{handler-seq}\br
    handler \opt{handler-seq}
\end{bnf}

\begin{bnf}
\nontermdef{handler}\br
    \terminal{catch (} exception-declaration \terminal{)} compound-statement
\end{bnf}

\begin{bnf}
\nontermdef{exception-declaration}\br
    \opt{attribute-specifier-seq} type-specifier-seq declarator\br
    \opt{attribute-specifier-seq} type-specifier-seq \opt{abstract-declarator}\br
    \terminal{...}
\end{bnf}

\begin{bnf}
\nontermdef{noexcept-specifier}\br
    \terminal{noexcept} \terminal{(} constant-expression \terminal{)}\br
    \terminal{noexcept}\br
\end{bnf}

\gramSec[gram.cpp]{Preprocessing directives}

\begin{bnf}
\nontermdef{preprocessing-file}\br
    \opt{group}
\end{bnf}

\begin{bnf}
\nontermdef{group}\br
    group-part\br
    group group-part
\end{bnf}

\begin{bnf}
\nontermdef{group-part}\br
    control-line\br
    if-section\br
    text-line\br
    \terminal{\#} conditionally-supported-directive
\end{bnf}

\begin{bnf}\obeyspaces
\nontermdef{control-line}\br
    \terminal{\# include} pp-tokens new-line\br
    \terminal{\# define } identifier replacement-list new-line\br
    \terminal{\# define } identifier lparen \opt{identifier-list} \terminal{)} replacement-list new-line\br
    \terminal{\# define } identifier lparen \terminal{... )} replacement-list new-line\br
    \terminal{\# define } identifier lparen identifier-list \terminal{, ... )} replacement-list new-line\br
    \terminal{\# undef  } identifier new-line\br
    \terminal{\# line   } pp-tokens new-line\br
    \terminal{\# error  } \opt{pp-tokens} new-line\br
    \terminal{\# pragma } \opt{pp-tokens} new-line\br
    \terminal{\# }new-line
\end{bnf}

\begin{bnf}
\nontermdef{if-section}\br
    if-group \opt{elif-groups} \opt{else-group} endif-line
\end{bnf}

\begin{bnf}\obeyspaces
\nontermdef{if-group}\br
    \terminal{\# if     } constant-expression new-line \opt{group}\br
    \terminal{\# ifdef  } identifier new-line \opt{group}\br
    \terminal{\# ifndef } identifier new-line \opt{group}
\end{bnf}

\begin{bnf}
\nontermdef{elif-groups}\br
    elif-group\br
    elif-groups elif-group
\end{bnf}

\begin{bnf}\obeyspaces
\nontermdef{elif-group}\br
    \terminal{\# elif   } constant-expression new-line \opt{group}
\end{bnf}

\begin{bnf}\obeyspaces
\nontermdef{else-group}\br
    \terminal{\# else   } new-line \opt{group}
\end{bnf}

\begin{bnf}\obeyspaces
\nontermdef{endif-line}\br
    \terminal{\# endif  } new-line
\end{bnf}

\begin{bnf}
\nontermdef{text-line}\br
    \opt{pp-tokens} new-line
\end{bnf}

\begin{bnf}
\nontermdef{conditionally-supported-directive}\br
    pp-tokens new-line
\end{bnf}

\begin{bnf}
\nontermdef{lparen}\br
    \descr{a \terminal{(} character not immediately preceded by white-space}
\end{bnf}

\begin{bnf}
\nontermdef{identifier-list}\br
    identifier\br
    identifier-list \terminal{,} identifier
\end{bnf}

\begin{bnf}
\nontermdef{replacement-list}\br
    \opt{pp-tokens}
\end{bnf}

\begin{bnf}
\nontermdef{pp-tokens}\br
    preprocessing-token\br
    pp-tokens preprocessing-token
\end{bnf}

\begin{bnf}
\nontermdef{new-line}\br
    \descr{the new-line character}
\end{bnf}

\begin{bnf}
\nontermdef{defined-macro-expression}\br
    \terminal{defined} identifier\br
    \terminal{defined (} identifier \terminal{)}
\end{bnf}

\begin{bnf}
\nontermdef{h-preprocessing-token}\br
    \textnormal{any \grammarterm{preprocessing-token} other than \terminal{>}}
\end{bnf}

\begin{bnf}
\nontermdef{h-pp-tokens}\br
    h-preprocessing-token\br
    h-pp-tokens h-preprocessing-token
\end{bnf}

\begin{bnf}
\nontermdef{has-include-expression}\br
    \terminal{\xname{has_include} ( <} h-char-sequence \terminal{> )}\br
    \terminal{\xname{has_include} ( "} q-char-sequence \terminal{" )}\br
    \terminal{\xname{has_include} (}   string-literal  \terminal{)}\br
    \terminal{\xname{has_include} ( <} h-pp-tokens     \terminal{> )}
\end{bnf}

\begin{bnf}
\nontermdef{has-attribute-expression}\br
    \terminal{\xname{has_cpp_attribute} (} pp-tokens \terminal{)}
\end{bnf}

