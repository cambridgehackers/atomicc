\infannex{gram}{Grammar integration with C++ summary}

\pnum
Summary of C++ grammar

\rSec1[gram.key]{Keywords}

\pnum
\indextext{keyword}%
New context-dependent keywords are introduced into a program by
\tcode{typedef}\iref{dcl.typedef},
\tcode{namespace}\iref{namespace.def},
class\iref{class}, enumeration\iref{dcl.enum}, and
\tcode{template}\iref{temp}
declarations.

\begin{ncbnf}
typedef-name:\br
	identifier
\end{ncbnf}

\begin{ncbnf}
namespace-name:\br
	identifier\br
	namespace-alias

namespace-alias:\br
	identifier
\end{ncbnf}

\begin{ncbnf}
class-name:\br
	identifier\br
	simple-template-id
\end{ncbnf}

\begin{ncbnf}
enum-name:\br
	identifier
\end{ncbnf}

\begin{ncbnf}
template-name:\br
	identifier
\end{ncbnf}

Note that a
\grammarterm{typedef-name}
naming a class is also a
\grammarterm{class-name}\iref{class.name}.

\input{std-gram-from-cpp.tex}
