%!TEX root = std.tex
\rSec0[dcl.dcl]{Declarations}%

\gramSec[gram.dcl]{Declarations}

\pnum
Declarations generally specify how names are to be interpreted. Declarations have
the form

\rSec2[atomicc.bitdecl]{integer bit width: __int(A)}

\begin{bnf}
\nontermdef{bit-type-specifier}\br
    \terminal{__uint} \terminal{(} constant-expression \terminal{)} \br
    \terminal{__int} \terminal{(} constant-expression \terminal{)}
\end{bnf}

\pnum
Function definitions have the form

\begin{bnf}
\nontermdef{atomicc-method-definition}\br
    \opt{decl-specifier-seq} interface-qualifier-seq identifier parameters-and-qualifiers function-body
\end{bnf}

\begin{bnf}
\nontermdef{interface-qualifier}\br
    identifier \terminal{.}

\nontermdef{interface-qualifier-seq} \br
    interface-qualifier \br
    interface-qualifier-seq interface-qualifier
\end{bnf}

\begin{bnf}
\nontermdef{atomicc-function-body}\br
    \opt{ctor-initializer} \opt{if-guard} compound-statement
\end{bnf}

