%!TEX root = std.tex
\rSec0[dcl.xxdcl]{xxDeclarations}%
\indextext{declaration|(}

\gramSec[gram.xxdcl]{xxDeclarations}

\indextext{linkage specification|see{specification, linkage}}

\pnum
Declarations generally specify how names are to be interpreted. Declarations have
the form

\begin{bnf}
\nontermdef{xxsimple-type-specifier}\br
    \opt{nested-name-specifier} type-name\br
    nested-name-specifier \terminal{template} simple-template-id\br
    \opt{nested-name-specifier} template-name\br
    \terminal{char}\br
    \terminal{char16_t}\br
    \terminal{char32_t}\br
    \terminal{wchar_t}\br
    \terminal{bool}\br
    \terminal{short}\br
    \terminal{int}\br
    \terminal{long}\br
    \terminal{signed}\br
    \terminal{unsigned}\br
    \terminal{float}\br
    \terminal{double}\br
    \terminal{void}\br
    \terminal{auto}\br
    decltype-specifier\br
    \terminal{__uint} \terminal{(} constant-expression \terminal{)} \atomicc \br
    \terminal{__int} \terminal{(} constant-expression \terminal{)} \atomicc
\end{bnf}

\pnum
\indextext{body!function}%
Function definitions have the form

\indextext{\idxgram{function-definition}}%
%
\begin{bnf}
\nontermdef{xxfunction-definition}\br
    \opt{decl-specifier-seq} interface-qualifier-seq identifier parameters-and-qualifiers function-body \atomicc\br
    \opt{attribute-specifier-seq} \opt{decl-specifier-seq} declarator \opt{virt-specifier-seq} function-body\br
    \opt{attribute-specifier-seq} \opt{decl-specifier-seq} declarator requires-clause function-body
\end{bnf}

\begin{bnf}
\nontermdef{xxinterface-qualifier} \atomicc\br
    identifier \terminal{.} \atomicc

\nontermdef{xxinterface-qualifier-seq} \atomicc\br
    interface-qualifier \atomicc\br
    interface-qualifier-seq interface-qualifier \atomicc
\end{bnf}

\begin{bnf}
\nontermdef{xxfunction-body}\br
    \opt{ctor-initializer} \opt{if-guard} compound-statement \atomicc\br
    function-try-block\br
    \terminal{= default ;}\br
    \terminal{= delete ;}
\end{bnf}

