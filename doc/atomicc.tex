%!TEX root = std.tex
\rSec0[atomicc.basic]{Basic}

\rSec1[atomicc.background]{Background}

Moore's Law improvements in manufacturing
costs of hardware have driven dramatic increases in the variety and complexity
of hardware systems that can be economically manufactured.
Unfortunately, high costs of producing verified hardware designs limit
deployment to devices that
can amortize development costs over very high manufacturing volumes.
We have seen dramatic improvements in general purpose processors
and memory devices, but much more limited improvements in application
specific devices.

Deploying new versions of hardware designs can be quite expensive,
placing extreme emphasis on gaining confidence of correct operation before
manufacturing, causing the costs of testing to
dominate product development costs.
How can we make the development of 
\textit{verified} designs more efficient and predictable?

The key to production of correct designs is the ability to guarantee
safety (nothing "bad" happens throughout execution)
and liveness (something "good" eventually does happen) of the design.
If we can guarantee each of these aspects at compile-time,
we significantly reduce the dependency on runtime verification.

\rSec1[atomicc.approach]{Approach}

To reduce verification costs, we propose:
\begin{itemize}
\item Module reuse without reverification.
\begin{itemize}
\item A single Verilog module is generated for each defined AtomicC module.
The generated Verilog is highly readable, giving designers confidence that
intended operations are correctly realized.
\item Module schedules calculated using only local information and are independent of module usage context.
\item Parameterized generated code: propagation of module parameterization from
source through to generated Verilog allows reuse without regeneration for each parameter instance.
\item Stable generated Verilog: To minimize the reverification effort for ECOs,
"local AtomicC source code changes (generally) should only cause local changes to generated Verilog".

In hardware design, the design phase is followed by significant validation effort
to ensure manufacturability.  Since the number of system parameters is
large, there is an extreme need to keep all unrelated areas of a design constant
when implementing engineering change orders to fix problems descovered
in the manufacturing process.
By modular generation of Verilog code and by enforced runtime isolation
of modules, ECOs are independent.

\end{itemize}

\item Compiler guarantees program runtime safety and (where possible) liveness
\begin{itemize}
\item Through HDL structuring, guarantee the absence of dynamic concurrency interleaving errors in generated Verilog.
\item When rules conflict, the compiler synthesizes fair arbitration, guaranteeing liveness between rules.
\item To guarantee liveness of user code, assert statements must be written by the user.
When possible, the compiler attempts to verify them with model checking or proof tools.
Since designing effective assert statements can be difficult, a standardized runtime
is provided allowing access to assertion failures in synthesized logic as well
as providing access to program trace information.
\end{itemize}

\item Standardized debug runtime
\begin{itemize}
\item Display and trace runtime support is provided, easing unit testing
as modules are created.
\item When not possible to statically verify asserts, a standard runtime exists for reporting errors.
\item hardware/software interactions are facilitated by a NOC DMA to processor memory,
smoothing interactions between high clock speed/narrow thread software and low clock
speed/wide thread hardware.
\item runtime support for scan-chain based testing
\end{itemize}
\item Integration with existing workflow
\begin{itemize}
\item Existing Verilog modules can be called from and can call AtomicC generated modules.
\item Standard Verilog backend tools are used to synthesize the resulting ASIC or FPGA.
\end{itemize}
\end{itemize}

\rSec1[atomicc.language]{AtomicC Overview}

AtomicC is a timed, structural hardware description language for
the high level specification of algorithms to be instantiated
directly in hardware.
AtomicC extends C++
with support for Guarded Atomic Actions
\cite{Hoe:Thesis,HoeArvind:TRSSynthesis2,Dave2007}:
Bluespec-style\cite{Bluespec:www}
modules, rules, interfaces, and methods.


The language is designed for
the construction of \textbf{modules} that are \textit{correct-by-construction composable}:
validated smaller modules can be aggregated to form
a larger validated module:
\begin{itemize}

\item Module interactions are performed with
latency insensitive
\textbf{method} calls, allowing methods to enforce invocation pre-conditions
and transitive support for stalling.

\item An \textbf{interface} is a named collection of method signatures and/or interface instances.
Modules can declare
multiple \textbf{interfaces}, giving each interface an explicit name,
giving flexibility in coupling with other modules.
Interfaces can be exported (defined in the module) or imported (used in
the module, but defined externally), giving flexibility in algorithm
representation.

\item All state elements in the hardware
netlist are explicit in the source code of the design.
All module data is private, externally accessable only by interface method invocation.

\item In AtomicC, user operations are written as SSA transactions, called \textbf{rules}.  Since the
compiler can statically analyze the read and write sets of the rule as well as the
invocation conditions, it can guarantee that the generated code always executes
in a Sequentially Consistent (SC) manner: concurrent execution of transactions is
guaranteed to be isolated.  For conflicts, a fair schedule is generated.

\item PSA support for assertions: assert, assume, restrict, cover.
Assertions can be either state-based (causal dependency between events) or
property-based (global).
\begin{itemize}
\item Safety. A safety property asserts that nothing "bad" happens throughout execution
\begin{itemize}
\setlength\itemsep{-0.5em}
\item Refinement checking
\item Invariant:  Basis.  Inductive step.
\item Deadlock freedom
\item Reachability
\end{itemize}
\item Liveness. A liveness property asserts that something "good" eventually does happen.
\begin{itemize}
\setlength\itemsep{-0.5em}
\item Termination
\item Response
\item Request-response properties
\item real-time (system acts "in time")
\end{itemize}
\item Trace equivalence
\end{itemize}

\item As with Connectal\cite{king2015software}, AtomicC designs may include both hardware and
software components, using interfaces to specify type safe communication.
The AtomicC compiler generates the code and transactors to pass
arguments between hardware and software.

\end{itemize}

\rSec1[atomicc.modfuture]{Future work}

Multiple clock domains

Need to describe multi-cycle rules and pipelining.

Need to describe joining rules

Need to have a way to support sequencing of operations

For NOC: allocate NOC like clock tree, after physical placement known.  Can do 
method calls across NOC using credit.

Physical partitioning is used to separate design into separately synthesized pieces, connected using
"long distance" signalling.  Parallel synthesis; bitstreams combined.

Incremental place and route for changes:
\begin{itemize}
\item Automated physical partitioning of the design into locally clocked logic islands
connected through delay tolerant network on chip
\item Quick edit/compile/test cycles minimize the time between making a change and
seeing the effect, allowing the developer to remain "in context"
\item By physically partitioning the design into connected, placed regions, only the
logic block that has been actually changed needs to be recompiled.  Modular reuse through pre-placed modules
\end{itemize}

Timing closure -- simplifying retiming:
\begin{itemize}
\item timing closure issues
can be separated to 2 groups: interblock and intrablock.
By using self-timed interfaces intrablock, long timing closure paths can be eliminated
\item simple decoration of source expression trees can be used to generate
fifo pipelined implementation of transactions.
\end{itemize}

Need to have a way to support model checking (say 'module B is a behavioral description of module A')
Show example with diff eqn solver from Sharp thesis.

Coding FSMs as 'case' statements in Verilog (to integrate with verification tools).

