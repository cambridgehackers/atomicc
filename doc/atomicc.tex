%!TEX root = std.tex
\rSec0[atomicc.basic]{Basic}

\rSec1[atomicc.intro]{Introduction}

AtomicC is a timed, structural hardware description language, permitting
the high level specification of algorithms to be instantiated
directly in hardware.
AtomicC is structural in that all state elements in the hardware
netlist are explicit in the source code of the design.
AtomicC follows in the tradition of the Go language,
promoting the use of interfaces when performing composition and
coupling rather than using inheritance \cite{Pike2012}.
Separate compilation of modules promotes the safe use of pre-compiled, validated
libraries, improving productivity on large projects.

AtomicC extends C++
with Bluespec-style\cite{Bluespec:www,Hoe:Thesis,HoeArvind:TRS_Synthesis2}
modules, rules, interfaces, and methods.
AtomicC does not attempt to emulate the behavior of all C++ constructs in hardware.
Instead, it uses a subset of the C++ language to specify desired
conditional assignments to state elements,
performing these operations under an atomic rule-based execution model.
Transactions are "write" or "commit" actions.

Like Connectal, AtomicC designs may include both hardware and
software components, using interfaces to specify hardware/software communication
in a type safe way. The AtomicC compiler generates the code and transactors to pass
arguments between hardware and software.

The basic building block of AtomicC is the module declaration,
consisting of 3 parts:
\begin{itemize}
\item instantiation of state elements used internally by the module,
\item the module's interface for interacting with other modules,
\item 'rules', which group conditional assignment statements and method invocations into atomic transactions.
\end{itemize}

\rSec1[atomicc.interface]{Interfaces}
An interface is a list of method signatures, but no method bodies, defining
an abstract data type(ADT) \cite{Liskov74programmingwith} for the module.
Named interface declarations allow a module to expose
multiple, independent, heterogeneous ADTs for system construction.
This allows explicit denotation in the program text
of which ADTs are connected to and accessable from other state elements.

A module can export named interface elements, providing implementations of
the methods of each of these interfaces.  A module definition can also import
interface references, allowing the module to invoke methods defined
in an external module.

Named interfaces elements can be:
\begin{itemize}
\item referenced by the instantiator of the module,
\item forwarded transparently as an interface for the instantiating module,
\item 'connected' to an 'interface reference' of another module in the instantiating scope.
\end{itemize}

There are 2 types of methods:
\begin{itemize}
\item \textbf{Action method procedures} perform persistent operations on state elements
and do not have a return value.  A compiler generated \textbf{ready} signal
indicates when the callee is available to perform the method invocation.  A compiler generated
\textbf{valid} signal indicates that the caller wishes to perform the method invocation (which
takes place only when both \textbf{ready} and \textbf{valid} are high during the same clock cycle).
\item \textbf{Value method functions} provide transient access to module state element values along
with a compiler generated \textbf{ready} signal to indicate when that value is valid.
\end{itemize}

Instatiated modules are interacted with through their interface methods.  The parameters
to these methods are used to pass data to the instantiated module.
AtomicC uses a \textbf{valid/ready} handshake process\cite{Fletcher2009, AXISpec} to invoke methods,
giving both the invoker(master) and invokee(slave) the ability to control the progress of execution.
The master uses the \textbf{valid} signal
of the method to show when data is available; the slave uses the associated
\textbf{ready} signal to show when the data can be accepted.
The method invocation transaction succeeds only when
both \textbf{valid} and \textbf{ready} are HIGH.
\textbf{Ready} signalling from the called module allows it to prevent
scheduling of the calling transaction until all necessary elements are
available.

\rSec1[atomicc.modrule]{Rules}
In AtomicC, all enabled statements in all modules execute on every clock cycle
as atomic transactions: internal states in the rule
execution are not visible to other rules.
The compiler synthesizes control signals that allow rules to fire
only when their dependant elements are ready and checks that enabled
rules in a cycle are
"equivalent" to some legal sequential computation \cite{Lamport:1979:MMC:1311099.1311750}.

The compiler statically determines the 'read' and 'write'
footprint for every rule.
If static compile analysis cannot prove that their execution is disjoint
or the source text of the program must specify the priority order of execution
when both rules are valid,
then two rules are not serializable (they have a 'read' or a 'write' conflict).

The AtomicC compiler verifies that it is valid to consider
all rules executed during a given clock cycle
as if they were serialized into a linear, atomic ordering ("sequentially consistent"(SC)).
Even though all concurrent rules are executed during the same clock cycle, SC allows us to compute the
outcome of each rule independantly of any other rules that could be executing at the same time.

\rSec1[atomicc.schedule]{Scheduling}

Each rule has a set of state elements that it reads and another set of element that it writes.
For the execution of a group of rules to be considered to be SC, the following must
be true:

\begin{itemize}
\item Atomic: All read and write operations for a given rule occur at the same time point in the sequence.
\item Read-before-write:  A rule that writes a state element must occur later in the sequence
than any rules that read the same state element.
\item Non-conflicting: A given state element cannot be written by more than one concurrently executable rule.
\end{itemize}

The compiler and linker do not break SC violations automatically. Errors require the user to annotate
the source text with "priority" statements to resolve conflicts.

\rSec2[atomicc.scheddefs]{Definitions}
\begin{itemize}
\item Rules: $R_{i}$
\item Methods: $M_{i}$
\item Control signals:
\begin {itemize}
\item Exported signal(generated by callee):  $ready(M_{i})$
\item Imported signal(generated by caller):  $valid(M_{i})$
\end{itemize}
\item Rule firing condition: $\pi(M_{i}) \equiv ready(M_{i})\ \&\&\ valid(M_{i})$
\item read set: $R_{i}.read$
\item write set: $R_{i}.write$
\item sensitivity set: $S(R_{i}) \equiv R_{i}.read \cup R_{i}.write$
\item arc condition: $arcCond(uv) \equiv
\newline
      fold (|, \{ \forall E \in u.Write, E \in v.Read \colon (u.Write(E).cond\ \&\ v.Read(E).cond) \})$
\end{itemize}
\rSec2[atomicc.schedalg]{Algorithm}

\begin{lstlisting}[mathescape=true]
$WorkSet$ = { all rules and methods for module }
for $r \in WorkSet$ do
    // Greedily group all rules/methods that have some overlap in read/write sets
    $V = \{ r \}$
    $V = \{ E \vert S(E) \cap S(V) \neq \varnothing  \}$  // Create next schedule set
    $WorkSet = WorkSet \setminus V$

    // Create 'read-before-write' digraph
    $G = (V, E)$
    where: 
       $V$ is a set of vertices
       $E$ is a set of arcs $==$ { uv $\vert$ arcCond(uv) is not identically false }

    // Find loops
    $L$ = loops in $G$

    for $L \in G$ do
       if fold (&, { $\forall uv \in L \colon$ arcCond(uv)}) is not identically false
          // 'break' loop L
          if loop has some method $M$ & some rule $R$
              $ready(R)$ &= $\neg \pi(M)$
          else if source code has "priority $R1 > R2$" & $R1 \in L$ & $R2 \in L$
              $ready(R2)$ &= $\neg \pi(R1)$
          else
              loop still exists, report error
\end{lstlisting}

\rSec2[atomicc.schedfuture]{Future directions in scheduling}

It is possible to create new rules that 'read' the 'commit value' for a state element.
This would be done by synthesising a new 'combined rule' and scheduling it \cite{Rosenband:Thesis}.

\rSec2[atomicc.schedprev]{Previous scheduling work}

In the Esposito Scheduler\cite{Esposito:Patent},
the Bluespec compiler creates a specific, linear schedule, adding
one rule at a time.  When a rule violates constraints from the previously scheduled
rules, an error is issued and an automatic guard is synthesised to prevent the
rule from executing on cycles when conflicting rules are executed.

Since the rules are added to the schedule approximately in the order they are
found in the source program text, the resulting schedule (and which rules are
inhibited) can be affected by source ordering and edits.

\rSec1[atomicc.modcomp]{Compilation}

The AtomicC compiler generates a separate Verilog module definition
for each source AtomicC module definition.
This verilog source defines the state elements
used in the design as well as their connections (netlist).

Modules independently compiled.  Combined with "linking", which validates schedule using header files.

Physical partitioning is used to separate design into separately synthesized pieces, connected using
"long distance" signalling.  Parallel synthesis; bitstreams combined.

AtomicC execution consists of 3 phases:
\begin{itemize}
\item static elaboration: netlist generation, 
\item netlist compilation or implementation
\item and runtime.
\end{itemize}

During netlist
generation, modules are instantiated by executing their
constructors. During this phase, any C++ constructs may be used, but
the resulting netlist must only contain synthesizeable components.

During netlist compilation, the netlist is analyzed and translated to
an intermediate representation and then to Verilog for simulation or
synthesis. Alternate translations are possible: to native code via
LLVM, to System C, to Gallina for formal verification with the Coq
Proof Assistant, etc.
