%!TEX root = std.tex
\rSec0[atomicc.basic]{Basic}

\rSec1[atomicc.intro]{Introduction}

AtomicC is a timed, structural hardware description language for
the high level specification of algorithms to be instantiated
directly in hardware.
AtomicC extends C++
with support for Guarded Atomic Actions:
Bluespec-style\cite{Bluespec:www,Hoe:Thesis,HoeArvind:TRS_Synthesis2}
modules, rules, interfaces, and methods.
AtomicC does not attempt to emulate the behavior of all C++ constructs in hardware,
instead uses a subset of the C++ language to specify behavioral
assignments to state elements.

The language is designed for
the construction of \textbf{modules} that are correct-by-construction \textit{composable}:
validated smaller modules can be aggregated to form
a larger validated module with no impact on the correct
operation of the component modules:
\begin{itemize}
\item Modules interactions are performed with Latency Insensitive (LI)
\cite{carloni2001theory}
\textbf{method} calls, allowing methods to enforce invocation pre-conditions
and transitive support for stalling.
\item Module behaviorial statements are encapsulated into transactions (\textbf{rules})
with full compile time validation of concurrent execution.
\cite{NikhilSemantics}
\cite{harris2005composable}.
\cite{nurvitadhi2011automatic}
\cite{fox2003algebraic}
\item An \textbf{interface} is a named collection of method signatures, defining
the behavior of an abstract data type(ADT) \cite{Liskov74programmingwith}.
Modules can export and import
multiple, named \textbf{interfaces},
giving flexibility in algorithm expression and coupling with other modules.
\item All state elements in the hardware
netlist are explicit in the source code of the design.
All module data is private to the module.
\end{itemize}
These features support the reliable reuse of pre-compiled, incrementally validated
libraries, improving productivity on large designs.

Like Connectal, AtomicC designs may include both hardware and
software components, using interfaces to specify hardware/software communication
in a type safe manner. The AtomicC compiler generates the code and transactors to pass
arguments between hardware and software.

The AtomicC compiler 
generates a single Verilog module for each AtomicC module declared.
Existing Verilog modules can be called from and can call AtomicC
generated modules.
Standard Verilog backend tools are used to synthesize
the resulting ASIC or FPGA.

The basic building block of AtomicC is the module declaration.
Modules consist of 3 parts:
\begin{itemize}
\item Instantiation of state elements used by the module,
\item Interface declarations for interacting from other modules,
\item Rules, which group assignment statements and method invocations into atomic transactions.
\end{itemize}

\rSec1[atomicc.interface]{Interfaces}

Exported interfaces can be used in several ways:
\begin{itemize}
\item invoked directly by the instantiator of the module,
\item forwarded transparently, becoming another exported interface of the instantiating module,
\item 'connected' to an 'interface reference' of another module in the instantiating scope.
\end{itemize}

There are 2 types of methods:
\begin{itemize}
\item \textbf{Value method functions} provide read-only access to module state elements.
\item \textbf{Action method procedures} perform write operations on state elements,
can take parameters and do not have return values.
A compiler generated
\textbf{valid} signal indicates that the caller wishes to perform the method invocation.
\end{itemize}

Both types of methods use a compiler generated \textbf{ready} signal
to indicate when the callee is available and
to prevent scheduling of the calling transaction until all necessary elements are
available.

AtomicC uses a \textbf{valid/ready} \textit{hand-shaking signalling}
\cite{Fletcher2009, AXISpec} to invoke action methods,
giving both the invoker(master) and invokee(slave) the ability to control transaction
execution timing.
The master uses the \textbf{valid} signal
of the method to show when parameter data is available and the transaction should be performed.
The method invocation transaction succeeds only when
both \textbf{valid} and \textbf{ready} are HIGH:

 ($\pi(M_{i}) \equiv ready(M_{i})\ \&\&\ valid(M_{i})$).

\rSec1[atomicc.modrule]{Rules}

Transactions are units of consistency (preserving program invariants), following ACID semantics.
In AtomicC, all enabled rules in all modules execute on every clock cycle.
The compiler synthesizes control signals, allowing rules to fire
only when their dependent elements and referenced method invocations are ready.

The AtomicC compiler
validates that all rules executed during a given clock cycle are
"sequentially consistent"(SC) \cite{Lamport:1979:MMC:1311099.1311750},
guaranteeing \textit{isolation}\cite[Sec.~7.1]{GrayR93}: each rule executes
independently of any other rules executing at the same time.

Of course, since all rules execute in a single cycle, "schedule" does not
refer to an actual time sequenced evolution, but merely to an abstract sub-cycle
virtual ordering.

In contrast to software, which uses \textit{dynamic allocation}\cite[p.~377]{GrayR93}
and locking\cite[Sec.~11.2]{OV11} to guarantee isolation,
the set of state elements accessed by an AtomicC transaction
and the boolean condition when the transaction is performed
are known at compile time.
This allows \textit{static allocation}\cite[Sec.~7.3.1]{GrayR93} of schedules.

In preparation for scheduling, the compiler statically determines the
\textit{read set} and \textit{write set}
for every rule and method \cite[Sec.~10.1.2]{OV11} \cite{RosenkrantzSternsLewis}.
The \textit{base set} is defined to be
the union of the \textit{read set} and \textit{write set}.

A \textit{constraint graph} is a partially-ordered digraph that models
the conditional dependencies within a set of rules.
\cite[Sec.~3]{Cain2003}.
The condition on each edge is the boolean expression for when the
edge (the dependency) exists, based on the conditions that each rule
and each local operation within that rule exists.
Nodes in a constraint graph represent atomic rule execution instances.  Edges
represent the ordering relations between these instances, and are transitive in nature.
The transitive closure of these orders on the constraint graph nodes
dictate the sequence in which each rule must \textit{appear} to execute in
order to be considered correct.
Dependency constraints are:
\begin{itemize}
\item \textbf{Read-after-write (RAW)}: two nodes are connected by RAW dependence
edges if one transaction creates a value that is used by the other, ensuring
that values are created before they are read.
\item \textbf{Write-after-read (WAR)}: the transaction node that writes a state
element is preceded by a node that reads that element.
\item \textbf{Write-after-write (WAW)}: the transaction node that writes a state
element is preceeded by another node that has previously written that element.
\end{itemize}
Currently in AtomicC,
since all reads of state elements occur at the beginning of a clock period
and all writes occur at the end,
only WAR edges are permitted in the constraint graph.
In the future, it is possible that RAW edges can be removed using value prediction
and WAW or RAW edges could be removed with register renaming, allowing more
flexible aggregation of multiple transactions into a single cycle.

To check that a set of rules is SC, 
first greedily partition the rules into \textit{schedule sets}: rules that have
overlapping \textit{base sets}.

To check that a 'schedule set' is SC, first
define a dependency digraph with node for each transactions and
edges from loads to stores, including the conditional execution conditions.
The transitive closure of these orders on the dependency graph nodes dictate
the sequence in which each operation must \textit{appear} to execute in order to
be considered correct.
\cite[Sec.~11.1]{OV11}

Each edge has an \textit{unit edge condition}, which is the disjunction of the following:
the execution condition of the source rule,
the execution condition of the destination rule,
the execution condition of the source read/write of the dependant state element,
the execution condition of the destination read/write of the dependant state element.

For each pair of nodes in the dependency digraph, we define the \textit{edge condition}
between 2 nodes as the 'fold' of the \textit{unit edge conditions} of each WAR edge
using the 'or' operator.
For a cycle in the digraph, we define the \textit{path condition}
as the 'fold' of the \textit{edge conditions} for all edges in the cycle using the 'and' operator.

If a loop has a \textit{path condition} that is not identically false,
the compiler can break the loop under the following conditions:
\begin{itemize}
\item if loop has some method $M$ \& some rule $R$

              $valid(R)$ \&= $\neg valid(M)$
\item if source code has "priority $R1 > R2$" \& $R1 \in L$ \& $R2 \in L$

              $valid(R2)$ \&= $\neg valid(R1)$
\end{itemize}
A \textit{schedule set} is not SC if the dependency digraph has a cycle
with a cycle \textit{path condition} that is not identically false.

The compiler and linker do not resolve SC conflicts automatically.
If static compile analysis cannot prove that their execution conditions(guards) or
effects(base sets) are always disjoint,
then the source text of the program must specify the priority order of scheduling
to resolve conflicts.

\rSec2[atomicc.schedlink]{Linking}

Without knowledge of the internals of a method, it must be assumed that all "action method"
calls to a state element conflict.  In addition, it must be assumed that all "value method" calls
must preceed all "action method" calls in any clock cycle.

Since method/method conflicts in a module cannot be validated in the absence of
information about their usage, this processing is delayed until the "module group binding"
stage of linking.
(It is not possible to resolve these conflicts standalone in the instantiated module.)

The linker cannot break any loops, but can only report on errors that are non-DAG.

Memoize checked results.

\rSec2[atomicc.schedfuture]{Future directions in scheduling}

It is possible to create new rules that 'read' the 'commit value' for a state element.
This would be done by synthesising a new 'combined rule' and scheduling it \cite{Rosenband:Thesis}.

\rSec2[atomicc.schedprev]{Previous scheduling work}

In the Esposito Scheduler\cite{Esposito:Patent},
the Bluespec compiler creates a specific, linear schedule, adding
one rule at a time.  When a rule violates constraints from the previously scheduled
rules, an error is issued and an automatic guard is synthesised to prevent the
rule from executing on cycles when conflicting rules are executed.

Since the rules are added to the schedule approximately in the order they are
found in the source program text, the resulting schedule (and which rules are
inhibited) can be affected by source ordering and edits.

\rSec1[atomicc.modcomp]{Compilation}

The AtomicC compiler generates a separate Verilog module definition
for each source AtomicC module definition.
This verilog source defines the state elements
used in the design as well as their connections (netlist).

Modules independently compiled.  Combined with "linking", which validates schedule using header files.

Physical partitioning is used to separate design into separately synthesized pieces, connected using
"long distance" signalling.  Parallel synthesis; bitstreams combined.

AtomicC execution consists of 3 phases:
\begin{itemize}
\item static elaboration: netlist generation, 
\item netlist compilation or implementation
\item and runtime.
\end{itemize}

During netlist
generation, modules are instantiated by executing their
constructors. During this phase, any C++ constructs may be used, but
the resulting netlist must only contain synthesizeable components.

During netlist compilation, the netlist is analyzed and translated to
an intermediate representation and then to Verilog for simulation or
synthesis. Alternate translations are possible: to native code via
LLVM, to System C, to Gallina for formal verification with the Coq
Proof Assistant, etc.

\rSec1[atomicc.modfuture]{Future work}

Need to describe multi-cycle rules and pipelining.

Need to have a way to support sequencing of operations

Need to have a way to support model checking (say 'module B is a behavioral description of module A')
Show example with diff eqn solver from Sharp thesis.

C block semantics do not correctly process the 2 statements: a = b; b = a;.
(binding of read values should occur at beginning of block, so that it is clear the
2nd assign refers to the 'previous' value).
Thinking again: if we retain C semantics, we have: temp = a; a = b; b = temp;, which
gives the correct value mapping.

Multiple clock domains
