%!TEX root = std.tex
\rSec0[atomicc.basic]{Basic}

\rSec1[atomicc.intro]{Introduction}
AtomicC is a structural hardware description language that extends C++
with Bluespec-style modules, rules, interfaces, and methods.

AtomicC is structural in that all state elements in the hardware
netlist are explicit in the source code of the design. AtomicC is a
timed HDL, using SystemC terminology. 
Atomic actions (rules and method invocations) execute in a single clock cycle.

Like Connectal, AtomicC designs may include both hardware and
software, using interfaces to specify hardware/software communication
in a type safe way. The AtomicC compiler generates the code to pass
arguments between hardware and software.

\rSec1[atomicc.compilation]{Compilation}

AtomicC execution consists of 3 phases:
\begin{itemize}
\item netlist generation, 
\item netlist compilation or implementation
\item and runtime.
\end{itemize}

During netlist
generation, modules are instantiated by executing their
constructors. During this phase, any C++ constructs may be used, but
the resulting netlist may only contain synthesizeable components.

During netlist compilation, the netlist is analyzed and translated to
an intermediate representation and then to Verilog for simulation or
synthesis. Alternate translations are possible: to native code via
LLVM, to System C, to Gallina for formal verification with the Coq
Proof Assistant, etc.

\rSec1[atomicc.schedule]{Scheduling}

Unlike C++'s serialized execution model, AtomicC allows all 'ready' rules (transactions) to be executed on
every clock cycle.  The AtomicC compiler verifies that it is possible to consider
all rules executed during a given clock cycle
as though they were serialized into some linear, atomic ordering (in other words, "sequentially consistent"(SC)).
Even though all concurrent rules are executed during the same clock cycle, SC allows us to examine the
effects of each rule independant of any rules that could be executing at the same time.

Each rule has a set of state elements that it reads and another set of element that it writes.
For the execution of a group of rules to be considered to be "sequentially consistent", the following must
be true:

\begin{itemize}
\item Atomic: All operations for a given rule occur at the same point in the sequence.
\item Read-before-write:  A rule that writes a state element must occur later in the sequence
than any rules that read the same state element.
\item Non-conflicting: A given state element cannot be written by more than one concurrently executing rule.
\end{itemize}

Scheduling is done by building a graph:
\begin{itemize}
\item Nodes are rules/methods within a module.
\item For all state elements, insert a directed links from each node that writes the state element to every node that reads it.
\end{itemize}

\begin{importgraphic}
{Simple ordering example}
{fig:schedule1}
{figschedule1.pdf}
\end{importgraphic}
A simple SC example is shown in Figure~\ref{fig:schedule1}.  There are 3 rules: RuleA, RuleB and RuleC.  There are 3 state elements: E1, E2 and E3.

\begin{importgraphic}
{non-SC ordering example}
{fig:schedule2}
{figschedule2.pdf}
\end{importgraphic}
A non-SC example is shown in Figure~\ref{fig:schedule2}.

Cycles can be broken in 2 ways:
\begin{itemize}
\item Rules default to have lower priority than methods within a module.
\item In the source text, "priority" statements can be used to break cycles if necessary.
\end{itemize}

