%!TEX root = std.tex
\rSec0[atomicc.basic]{Basic}

\rSec1[atomicc.intro]{Introduction}
AtomicC is a structural hardware description language that extends C++
with Bluespec-style\cite{Bluespec:www,Hoe:Thesis,HoeArvind:TRS_Synthesis2}
modules, rules, interfaces, and methods.
The compiler automatically synthesizes control signals that allow rules to fire
only when their dependant elements are ready and when there are no conflicts with
other rules executing in the same cycle.  The removal of this analytic burden on
the engineer increases productivity as well as improves reliability of the
resulting design.

AtomicC is structural in that all state elements in the hardware
netlist are explicit in the source code of the design. AtomicC is a
timed HDL, using SystemC terminology. 
Atomic actions (rules and method invocations) execute in a single clock cycle.
AtomicC does not attempt to replicate the behavior of all C++ constructs in hardware.
Instead, it uses the C++ text to specify the necessary single static assignment(SSA) computations
and performing these computations under an atomic rule-based execution model.
In addition, an interface definition scheme is used to explicitly and flexibly
manage the visibility of interface methods to a module.

Unlike C++'s serialized execution model, AtomicC allows all firing rules to be atomically executed on
every clock cycle.  The AtomicC compiler verifies that it is valid to consider
all rules executed during a given clock cycle
as if they were serialized into a linear, atomic ordering ("sequentially consistent"(SC)).
Even though all concurrent rules are executed during the same clock cycle, SC allows us to compute the
outcome of each rule independantly of any other rules that could be executing at the same time.

Like Connectal, AtomicC designs may include both hardware and
software, using interfaces to specify hardware/software communication
in a type safe way. The AtomicC compiler generates the code to pass
arguments between hardware and software.

\rSec1[atomicc.compilation]{Compilation}

AtomicC execution consists of 3 phases:
\begin{itemize}
\item netlist generation, 
\item netlist compilation or implementation
\item and runtime.
\end{itemize}

During netlist
generation, modules are instantiated by executing their
constructors. During this phase, any C++ constructs may be used, but
the resulting netlist may only contain synthesizeable components.

During netlist compilation, the netlist is analyzed and translated to
an intermediate representation and then to Verilog for simulation or
synthesis. Alternate translations are possible: to native code via
LLVM, to System C, to Gallina for formal verification with the Coq
Proof Assistant, etc.

\rSec1[atomicc.schedule]{Scheduling}

Each rule has a set of state elements that it reads and another set of element that it writes.
For the execution of a group of rules to be considered to be SC, the following must
be true:

\begin{itemize}
\item Atomic: All operations for a given rule occur at the same point in the sequence.
\item Read-before-write:  A rule that writes a state element must occur later in the sequence
than any rules that read the same state element.
\item Non-conflicting: A given state element cannot be written by more than one concurrently executing rule.
\end{itemize}

The compiler and linker do not break SC violations automatically. Error require the user to annotate
the source text with "priority" statements to resolve conflicts.

\rSec2[atomicc.scheddefs]{Definitions}
\begin{itemize}
\item Rules: $R_{i}$
\item Methods: $M_{i}$
\item Control signals:
\begin {itemize}
\item Exported signal(generated by callee):  $ready(M_{i})$
\item Imported signal(generated by caller):  $valid(M_{i})$
\end{itemize}
\item Rule firing condition: $\pi(M_{i}) \equiv ready(M_{i})\ \&\&\ valid(M_{i})$
\item read set: $R_{i}.read$
\item write set: $R_{i}.write$
\item sensitivity set: $S(R_{i}) \equiv R_{i}.read \cup R_{i}.write$
\item schedule set: all rules that could possibly conflict (rules that share elements in sensitivity set)
\end{itemize}

\rSec2[atomicc.schedalg]{Algorithm}

\begin{lstlisting}[mathescape=true]
// Partition rules into disjoint "schedule sets"
U = { $R_{i}, M_{i}$ forall i }   // Construct set of unscheduled rules
While $U \neq \emptyset$           // While there are unscheduled rules
    Extract a rule, T, from U
    $P_{i} = \{ T \}$          // Create next schedule set
    forall E in U
        if $S(E) \cap S(P_{i}) \neq \emptyset$
            Move E from U to $P_{i}$

forall $P_{i}$
    // Create 'read-before-write' graph
    Initialize graph G to have nodes for all elements in $P_{i}$
    forall T in $P_{i}$
        forall W in T.write
            forall J in $P_{i}$
                forall R in J.read
                    if W.name == R.name
                        add arc [$T \Rightarrow J$; guard: $(W.cond\ \&\&\ R.cond\ \&\&\ \pi(T)\ \&\&\ \pi(J))$]
    // Check/repair 'read-before-write' graph to be SC
    forall loops L in G
       loopcondition = true
       forall arcs A in L 
          loopcondition = loopcondition & A.guard
       if loopcondition is not identically false
          if loop has some method $M_{i}$ and some rule $R_{j}$
              $ready(R_{j}) = ready(R_{j})\ \&\&\ \neg \pi(M_{i})$
          else if source code has "priority $R_{i} > R_{j}$" & $R_{i}$ in L & $R_{j}$ in L
              $ready(R_{j}) = ready(R_{j})\ \&\&\ \neg \pi(R_{i})$
          else
              loop still exists, report error
\end{lstlisting}

