%!TEX root = std.tex
\rSec0[atomicc.modularization]{Modularization}

\rSec1[atomicc.independant]{Independant compilation of modules}

The design is separated into modules that can export and import interfaces to other modules.
Each source language module compiles into a single verilog module.  Modules are independantly
compiled, depending only on the interface definitions for referenced modules.
Referencing modules do not depend on the internal implementation of referenced modules,
even if they textually exist in the same compilation unit.
Scheduling of rules in a module is performed "inside out", with the resulting schedule dependancies written to a metadata file during compilation.

\rSec1[atomicc.econtrol]{Execution control}

There are 2 common styles for communication of execution control information for a method:
\begin{itemize}
\item Asymmetric (ready/enable signalling)
A method/rule is invoked by asserting the "enable" signal.  This signal can only be
asserted if the "ready" signal was valid, allowing the called module to restrict
permissible execution sequences.

\item Symmetric (ready/valid signalling)
Both caller/callee have "able to be executed" signals.  Execution is deemed to take
place in each cycle where both "ready" (from the callee) and "valid" (from the caller)
are asserted.
\end{itemize}

Bluespec uses the Asymmetric signalling style, collecting all scheduling control into a central location for analysis/generation.
AtomicC uses the Symmetric signalling style, giving modules local control over their allowable execution patterns.  Conflicts between local schedules for modules when they are connected together are detected by the linker.

\rSec1[atomicc.linker]{Linking of groups of modules}

To verify that an instantiated group of modules has SC compliant execution characteristics, a linker is used to cross check information from the metadata files for each module.

\rSec1[atomicc.verilog]{Interfacing with verilog modules}

To reference a module in verilog, fields can be declared in __interface items.
For example:

\begin{codeblock}
     __interface CNCONNECTNET2 {
         __input  __int(1)         IN1;
         __input  __int(1)         IN2;
         __output __int(1)         OUT1;
         __output __int(1)         OUT2;
     };
     __emodule CONNECTNET2 {
         CNCONNECTNET2 _;
     };
\end{codeblock}

This will allow references/instantiation of an externally defined verilog module CONNECTNET2
that has 2 'input' ports, IN1 and IN2, as well as 2 'output' ports, OUT1 and OUT2.

\rSec2[atomicc.param]{Parameterized modules}

Verilog modules that have module instantiation parameters can also be declared/referenced.
For example:

\begin{codeblock}
     __interface Mmcme2MMCME2_ADV {
         __parameter const char *  BANDWIDTH;
         __parameter float         CLKFBOUT_MULT_F;
         __input  __uint(1)        CLKFBIN;
         __output __uint(1)        CLKFBOUT;
         __output __uint(1)        CLKFBOUTB;
     };
     __emodule MMCME2_ADV {
         Mmcme2MMCME2_ADV _;
     };
\end{codeblock}

This example can be instantiated as:

\begin{codeblock}
     __module Test {
         ...
         MMCME2_ADV#(BANDWIDTH="WIDE",CLKFBOUT_MULT_F=1.0) mmcm;
         ...
         Test() {
            __rule initRule {
                mmcm._.CLKFBIN = mmcm._.CLKFBOUT;
            }
         }
     }
\end{codeblock}

\rSec2[atomicc.refsyntax]{Reference syntax}

For declaring ports in an interface:
\begin{codeblock}
     __interface <interfaceName> {
          __input/__output/__inout/__parameter <elementType> <elementName>;
     }
\end{codeblock}
For '__parameter' items, supported datatypes include: "const char *", "float", "int".

\rSec2[atomicc.ifactor]{Factoring of interfaces into sub interfaces is also supported}

\rSec2[atomicc.clockReset]{Clock/reset ports}
Note that if interface port pins are declared in a module interface declaration, then
CLK and nRST are _not_ automatically declared/instantiated.  (Since the user needs the
flexibility to not require them when interfacing with legacy code).

Note that this also allows arbitrary signals (like the output of clock generators) to be
passed to modules as CLK/nRST signals.  (For Atomicc generated modules, please note that the
default clock/reset signals for a module will always have these names)

\rSec2[atomicc.itool]{Import tooling}

There is a tool to automate the creation of AtomicC header files from verilog source files.
For example:
\begin{codeblock}
     atomiccImport -o MMCME2_ADV.h -C MMCME2_ADV -P Mmcme2 zynq.lib
     atomiccImport -o VMMCME2_ADV.h -C MMCME2_ADV -P Mmcme2 MMCME2_ADV.v
\end{codeblock}
