%!TEX root = std.tex
\rSec0[modules]{Modules}%

\gramSec[agram.class]{Classes}

The basic building block of AtomicC is the module declaration, made of 3 parts:
\begin{itemize}
\item Instantiation of state elements used by the module,
\item Interface declarations for interacting with other modules,
\item Rules, which group assignment statements and method invocations into atomic transactions.
\end{itemize}

\rSec1[atomicc.interface]{Module interface definition}

An \textbf{interface} is a named collection of method signatures
and/or interface instances.

There are 2 types of methods:
\begin{itemize}
\item \textbf{Value method functions} allow \textit{inspection} of module state elements.
\item \textbf{Action method procedures} allow \textit{modification}
(including inspection) of state elements,
can take parameters and do not have return values.
A compiler generated
\textbf{valid} signal indicates that the caller wishes to perform the method invocation.
\end{itemize}

Both value and action methods use a compiler generated \textbf{ready} signal
to indicate when the callee is available and
stall scheduling of the calling transaction until
execution pre-conditions are statisfied.

\begin{example}
\begin{codeblock}
     __interface EchoRequest {
         void say(__int(32) v);
         void say2(__int(16) a, __int(16) b);
     };
\end{codeblock}
\end{example}

\rSec1[atomicc.module]{Module declaration and definition}

Modules are independently compiled.
Rule and interface method scheduling logic is generated as part of the
generated module.  Scheduling constraints (read set, write set and relation to
other scheduled elements) are generated into a metadata file, allowing
schedule consistency checks between modules to be verified by the linker.

The design is separated into modules that can export and import interfaces to other modules.
Each source language module compiles into a single verilog module.  Modules are independently
compiled, depending only on the interface definitions for referenced modules.
Referencing modules do not depend on the internal implementation of referenced modules,
even if they textually exist in the same compilation unit.
Scheduling of rules in a module is performed "inside out", with the resulting schedule dependancies written to a metadata file during compilation.

Exported interfaces can be used in several ways:
\begin{itemize}
\item invoked directly by the instantiator of the module,
\item forwarded transparently, becoming another exported interface of the instantiating module,
\item 'connected' to an 'interface reference' of another module in the instantiating scope.
\end{itemize}

\begin{example}
\begin{codeblock}
     __module Echo {
         EchoRequest      request;               // exported interface (defined by this module)
         EchoIndication   *indication;           // imported interface (defined by the instantiator of this module)
         bool busy;
         __int(32) itemSay;
         ...
         // implementation of method request.say(). Note the guard "if (!busy)".
         void request.say(__int(32) v) if(!busy) {
             itemSay = v;
             ...
         }
         void request.saw(__int(16) a, __int(16) b) if(!busy) {
             ...
         }
     };
\end{codeblock}
\end{example}

To reference a module from a separate compilation unit, use "__emodule".  External
module definitions need only specify the exported/imported interfaces.

\begin{example}
\begin{codeblock}
     __emodule EchoResponder {
         EchoIndication   indication;           // exported interface
     };
\end{codeblock}
\end{example}

\rSec1[atomicc.guard]{Guard clauses on module interface methods}

\pnum
Method definitions in __module declarations have the form:

\begin{bnf}
\nontermdef{atomicc-method-definition}\br
    \opt{decl-specifier-seq} interface-qualifier-seq identifier parameters-and-qualifiers function-body
\end{bnf}

\begin{bnf}
\nontermdef{interface-qualifier}\br
    identifier \terminal{.}

\nontermdef{interface-qualifier-seq} \br
    interface-qualifier \br
    interface-qualifier-seq interface-qualifier
\end{bnf}

\begin{bnf}
\nontermdef{atomicc-function-body}\br
    \opt{if-guard} compound-statement

\nontermdef{if-guard}\br
    \terminal{if (} condition \terminal{)}

\end{bnf}


Rules are only ready to fire if the rule's guard is true and all the
guards on methods invoked within the rule are also true.

\begin{codeblock}
         void request.say(__int(32) v) if(!busy) {
             itemSay = v;
             ...
         }
\end{codeblock}

\rSec1[atomicc.poly]{Parametric Polymorphism}
The generated verilog can be parameterized by datatype bitsizes and array lengths.
To allow code generation independent of sizes,
source 'for' statements are translated into
Verilog generate 'for' loops.

\rSec1[atomicc.connect]{Connecting exported interfaces to imported references}
The __connect statement allows exported interface declarations to be connected
with imported interface references between objects within a module declaration.

\begin{bnf}
\nontermdef{connect-declaration}\br
    \terminal{__connect} identifier \terminal{=} identifier \terminal{;}
\end{bnf}

\begin{example}
\begin{multicols}{2}
AtomicC example
\begin{codeblock}
     __interface ExampleRequest {
         void say(__int(32) v);
     };

     __module A {
          ExampleRequest callIn;
     };

     __module B {
         ExampleRequest *callOut;
     };

     __module C {
         A consumer;
         B producer;
         __connect producer.callOut = consumer.callIn;
     };
\end{codeblock}
\columnbreak
BSV example
\begin{codeblock}
    BSV example
    BSV example
    BSV example
    BSV example
    BSV example
\end{codeblock}
\end{multicols}
\end{example}

Comparision with BSV:
\begin{itemize}
\item The declaration for 'A' is just like BSV.
In BSV, the declaration for B requires the interface instance for 'callOut' be passed in as
an interface parameter (forcing a textual ordering to the source code
declaration sequence).
\item In AtomicC, the interfaces are stitched together outside in any
convenient sequence in a location where both the concrete instances
for A and B are visible.
\end{itemize}

\rSec1[atomicc.export]{Exporting interfaces from contained objects}

In a design, there are times when the engineer wishes to declare an object locally,
but allow external modules to access specific interfaces of the local object.
This is done by declaring an interface to the containing object of compatible
type and just 'assigning' the local object's interface to it.

\begin{example}
\begin{codeblock}
     __module CWrapper {
         A consumer;
         ExampleRequest request = A.callIn;
      };
\end{codeblock}
\end{example}

CWrapper just forwards the interface 'request' down into the instance 'consumer'.

\rSec1[atomicc.classsyn]{Syntax extension to C++}
\begin{bnf}
\nontermdef{atomicc-class-key}\br
    \terminal{__interface} \br
    \terminal{__emodule} \br
    \terminal{__module}
\end{bnf}
