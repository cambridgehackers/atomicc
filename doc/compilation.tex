%!TEX root = std.tex
\rSec0[atomicc.modcomp]{Compilation}

AtomicC execution consists of the following phases:
\begin{itemize}
\item \textit{compilation}
\begin{itemize}
\item {Parsing,semantic checks},
\item {Static elaboration},

During \textit{static elaboration},
constructors are executed for statically declared data elements,
allowing allocation of new instances of modules and parameterized rule creation.
Any C++ constructs may be used, but
the resulting netlist must only contain synthesizeable components.

from Newton: 
Static elaboration does not change the semantics (types, evaluation rules) of the
Regiment language, it merely opportunistically pushes evaluation forward into compile time.

We need the ability to create state elements programmatically, improving
efficiency of the design creation process.
\item {Translation to an intermediate representation (IR)},
\item {Verilog netlist generation from the IR},
\end{itemize}
\item \textit{Linking}: Modules across the project are incrementally compiled
and unit tested.  As modules are reused in varying contexts, it is necessary
to validate if restrictions need to be added for concurrent calls to conflicting
methods in the module interface.
This checking is done by a \textit{linker},  ensuring that rule/method access
across a set of Verilog output is free of inter-module schedule conflicts,
\item \textit{Logic synthesis, physical backend processing}.  This is performed
with existing backend tool flows
\item \textit{Formal verification} using the Coq Proof Assistant, etc.
\item \textit{Verification}: performed using existing backend tooling,
\item \textit{Hardware execution}.
\end{itemize}

\begin{itemize}
\item generate Verilog
\item synthesize logic, get size
\item pack blocks with sizes
\item P \& R
\end{itemize}

\rSec1[atomicc.linker]{Linking of groups of modules}

To verify that an instantiated group of modules has SC compliant execution characteristics, a linker is used to cross check information from the metadata files for each module.
