%!TEX root = std.tex
\rSec0[stmt.stmt]{Statements}%

\gramSec[agram.stmt]{Statements}

AtomicC does not attempt to emulate the serialized execution
behavior of all C++ constructs in hardware.
Instead it uses a subset of the C++ language to specify
code blocks that have at most
one enabled assignment to any state element.
This form is called \textit{static single assignment} form,
or SSA form\cite{Alpern:1988}.
In the eventual runtime execution, these assignments are
all made in a single clock cycle, when the rule or method
is enabled.

Constant bound "for" statements that can be fully unrolled are supported.

Since AtomicC does not generate logic to orchestrate sequential execution behavior from
language constructs,
traditional C++ statements with non-static control flow behavior are not supported.

Examples include:
\begin{itemize}
\item Non-constant bound "for" statements.  
\item "do", "while" statements
\item Usages of "goto" that result in a cyclic directed graph of execution blocks
\item Method and function calls that are not inlinable at compilation time (for example, recursion is prohibited)
\end{itemize}


\rSec1[atomicc.rule]{__rule}

Rules specify a group of operations that must
execute transactionally: when a rule's guard
is satisfied, then it is ready to fire.

Module behaviorial statements are encapsulated into transactions (\textbf{rules})
following ACID semantics
\cite{NikhilSemantics,GrayR93};
all rules executed during a given clock cycle are
\textit{sequentially consistent} (SC) \cite{Lamport:1979:MMC:1311099.1311750},
guaranteeing each rule executes
independently of any other rules executing at the same time
\cite[Sec.~7.1]{GrayR93}.

\begin{bnf}
\nontermdef{rule-statement}\br
    \terminal{__rule} identifier \opt{if-guard} compound-statement\br

\end{bnf}

\begin{example}
\begin{codeblock}
     __rule respond_rule if (responseAvail) {
         fifo->out.deq();
         ind->heard(fifo->out.first());
     }
\end{codeblock}
\end{example}

\rSec1[atomicc.nostmt]{Model details}

C block semantics do not correctly process the 2 statements: a = b; b = a;.
(binding of read values should occur at beginning of block, so that it is clear the
2nd assign refers to the 'previous' value).
Thinking again: if we retain C semantics, we have: temp = a; a = b; b = temp;, which
gives the correct value mapping.

To preserve the standard interpretation of C++ source code
in methods and rules,
a \textbf{modification in-private} \cite[Sec.~3.2]{Prinz03operationalsemantics}
execution model is used:
\begin{enumerate}
\item Wrap each rule/method with prelude code and postprocessing code
\item Prelude code:
For each state element \textbf{A} in module, add the declaration of a shadow item:
\begin{codeblock}
decltype(this->A) A = this->A;    // create shadow of state element
\end{codeblock}
\item Postprocessing code:
For each state element actually written during execution of the code block:
\begin{codeblock}
this->A = A;                // update state elements only at end of code block
\end{codeblock}
All assignments in the postprocessing section occur on a single clock cycle.
\end{enumerate}
