\infannex{scheduleExample}{Scheduling example}

\pnum

\rSec1[orderExample.sw]{Source program}

User program source code:
\begin{codeblock}
__interface UserRequest {
    void say(__uint(32) va);
};

__module Order {
    UserRequest                     request;
    __uint(1) running;
    __uint(32) a, outA, outB, offset;
    void request.say(__uint(32) va) if (!running) {
        a = va;
        offset = 1;
        running = 1;
    }
    __rule A if (!__valid(request.say)) {
        outA = a + offset;
        if (running)
            a = a + 1;
    };
    __rule B if (!__valid(request.say)) {
        outB = a + offset;
        if (!running)
            a = 1;
    };
    __rule C if (!__valid(request.say)) {
        offset = offset + 1;
    };
};
\end{codeblock}


\rSec1[orderExample.verilog]{Verilog output}

This is the generated verilog:

\lstinputlisting[language=Verilog,basicstyle=\tiny\ttfamily]{examples/Order.v.tex}

\rSec1[orderExample.conflict]{Conflict graph}

%% In the following examples, there are 3 rules: RuleA, RuleB and RuleC.
%% There are 3 state elements: E1, E2 and E3.
%% 
\begin{importgraphic}
{Simple ordering example}
{fig:orderschedule}
{orderschedule.pdf}
\end{importgraphic}
%% 
%% A simple SC example is shown in Figure~\ref{fig:schedule1}.  
%% The schedule sequence is {A, B, C}.
%% 
%% \begin{importgraphic}
%% {non-SC ordering example}
%% {fig:schedule2}
%% {figschedule2.pdf}
%% \end{importgraphic}
%% 
%% A non-SC example is shown in Figure~\ref{fig:schedule2}.
%% There is no linear sequence of the rules that preserves read-before-write for all
%% state elements.
