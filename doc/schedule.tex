\Annex{scheduleDetail}{Scheduling Algorithm}{}

\rSec1[atomicc.schedgoal]{Goals}
To guarantee \textit{isolation} in transaction systems,
some form of concurrency control is necessary.
We can see a classification of concurrency control algorithms below (adapted from
\cite[Sec.~11.2]{OV11}):

\begin{importgraphic}
{Concurrency Control Algorithms}
{fig:concurrency}
{concurrency.jpg}
\end{importgraphic}

ADD TO PICTURE: Static schedule

To guarantee \textit{isolation} in the presence of
parallelism in software systems,
\textit{dynamic allocation}\cite[Sec.~7.3.1]{GrayR93} of schedules is used.
In hardware design with AtomicC,
the set of state elements accessed by a transaction,
the operations on these state elements (read-only or write)
and the boolean condition when the transaction is performed
are all known at compile time.
This allows static allocation of \textbf{schedules}
(sequences of transaction execution)
and compile time validation of SC.

Since no call by reference, all conflicts are "local conflicts".

\textbf{Should we just generate fair schedules in case of conflicts?}
\begin{itemize}
\item for write conflicts, use multiway RR arbiter
\item for SC conflicts, break cycle with an alternating boolean
\end{itemize}


\rSec1[atomicc.schedsignal]{Signalling}

AtomicC uses \textbf{valid/ready} \textit{hand-shaking signalling}
\cite{Fletcher2009, AXISpec} to invoke action methods,
giving both the invoker(master) and invokee(slave) the ability to control invocation
execution timing.
The master uses the \textbf{valid} signal
of an action method to show when parameter data is available
and the operation should be performed.
The method invocation succeeds only when
both \textbf{valid} and \textbf{ready} are HIGH in the same clock cycle.
\newline
In TRS notation\cite[p.~22]{Hoe:Thesis}:
\begin{itemize}[label= ]
\item $\pi(M_{i}) \equiv ready(M_{i})\ \wedge valid(M_{i})$.
\end{itemize}

\rSec1[atomicc.schedalg]{Algorithm}
The scheduling algorithm is:
\begin{itemize}
\item For each module, rules and methods that have some overlap of state element usage
(\textit{read set} and \textit{write set}\cite[Sec.~10.1.2]{OV11} \cite{RosenkrantzSternsLewis})
are greedily gathered into \textit{schedule sets}.
Since there can be no execution interactions between sets,
each set will be independently scheduled.
\item A \textit{constraint graph} is a partially-ordered directed graph modeling
the schedule sequencing dependencies within a \textit{schedule set}:
\begin{itemize}
\item \textit{nodes} in a constraint graph represent atomic rule and method instances,
\item \textit{edges} represent \textbf{write-after-read (WAR)} ordering dependency
for a specific storage element\cite[Sec.~3]{Cain2003}.

In addition, each edge has a symbolic boolean \textit{edge condition} for when the
the ordering dependency exists: the boolean
condition when one rule/method actually reads a given state element
and the other actually writes it.
\end{itemize}

\item The transitive closure of these orders on the constraint graph nodes
dictate the \textbf{schedule} in which each rule must \textit{appear} to execute in
order to be considered SC
\cite[Sec.~11.1]{OV11}.
Of course, since all rules execute in a single cycle, "schedule" does not
refer to an actual time sequenced evolution of state,
but to a \textit{abstract} "sub-cycle" ordering.

\item For each pair of nodes in the constraint digraph, we define the \textit{node condition}
between 2 nodes as the conjunction of the \textit{edge conditions} of all the edges between
them (i.e., the condition that \textit{any} of the edges causes a dependency).
For each cycle in the digraph, we define the \textit{path condition}
as the disjunction of the \textit{node conditions} for all sequential pairs of nodes
in the cycle (i.e., the condition that \textit{all} the edges, hence the cycle exists).

\item Since potential conflicts between methods (called from rules outside the module)
and module rules are quite common,
if cycle has some method $M$ \& some rule $R$, then the compiler can
automatically rewrite the term $valid(R)$ to add a disjunction with the term $\neg valid(M)$, breaking
the cycle.

\item When the \textit{path condition} is not identically false, a total ordering of
the digraph can not be guaranteed and the \textit{schedule set} is not SC.
In this case, the compiler or linker reports an error,
requiring resolution by the user.
\end{itemize}

A simple example of a constraint graph is given in \ref{scheduleExample}, at the end of this document.

Since AtomicC performs scheduling analysis independantly for each
declared module, external method invocation conflicts in rules cannot be validated.
Schedule processing for external method calls is delayed until the "module group binding"
stage of linking, where separately compiled AtomicC output is combined and
verified
for SC scheduling.  Errors and conflicts detected at this stage must be repaired
in the module source text and recompiled before proceeding.

\rSec1[atomicc.schedprev]{Previous scheduling work}

In Rule Composition\cite{Dave2007}, scheduling
is reformulated in terms of rule composition, leading to a succinct discussion
of issues involved, including a concise description of the Esposito and
Performance Guarantees schedulers.  The resulting schedules are quite close
to the user-specified scheduling in AtomicC.
In contrast to AtomicC, the Bluespec kernel language they use
for analysis also has a sequential composition operator, creating rules
that execute for multiple clock cycles.

The Esposito Scheduler\cite{Esposito:Patent,Dave2007},
is the standard scheduler generation algorithm in the Bluespec Compiler.
It uses a heuristic designed to produce a concrete total ordering of rules.

The Performance Guarantees scheduler\cite{Rosenband:PerformanceGuarantees}
was proposed to address issues with intra-cycle data passing.


\rSec1[scheduleExample]{Scheduling example}{}

\rSec2[orderExample.sw]{Source program}

\begin{codeblock}
__interface UserRequest {
    void say(__uint(32) va);
};

__module Order {
    UserRequest                     request;
    __uint(1) running;
    __uint(32) a, outA, outB, offset;
    void request.say(__uint(32) va) if (!running) {
        a = va;
        offset = 1;
        running = 1;
    }
    __rule A if (!__valid(request.say)) {
        outA = a + offset;
        if (running)
            a = a + 1;
    };
    __rule B if (!__valid(request.say)) {
        outB = a + offset;
        if (!running)
            a = 1;
    };
    __rule C if (!__valid(request.say)) {
        offset = offset + 1;
    };
};
\end{codeblock}


\rSec2[orderExample.conflict]{Constraint graph}

\begin{importgraphic}
{Simple ordering example}
{fig:orderschedule}
{orderschedule.pdf}
\end{importgraphic}

Sequentially consistent schedules are:
%%\setlist{nolistsep}
\begin{itemize}[noitemsep,topsep=0pt]
\item when 'running == 1': A -> B -> C
\item when 'running == 0': B -> A -> C
\end{itemize}

\rSec2[orderExample.verilog]{Verilog output}

\lstinputlisting[language=Verilog,basicstyle=\tiny\ttfamily]{examples/Order.v.tex}

\rSec2[orderExample.ir]{Intermediate representation output}

\lstinputlisting[language=Verilog,basicstyle=\tiny\ttfamily]{examples/generated/order.generated.IR}
