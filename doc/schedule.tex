\Annex{scheduleDetail}{Scheduling Algorithm}{}

\rSec1[atomicc.schedgoal]{Goals}

All changes to state elements are grouped into atomic transactions;
any execution trace is expressable as a linear sequence of isolated transaction invocations
(sequential consistent (SC)).
To execute transactions
concurrently within a single cycle,
we need to prove that concurrent execution behaves as though the transactions were
scheduled in some (imputed) linear sequence.

Write operation effects become visible only after the end of the clock
cycle, so there must be no read operations of the state element
in successive transactions within the same cycle (since they will observe "old" data).
We define a \textbf{write-after-read (WAR)} strict partial order \cite[Sec.~3]{Cain2003}
over the set of rule/method nodes in a module.
\cite[Sec.~10.1.2]{OV11} \cite{RosenkrantzSternsLewis}

There is a WAR constraint between rules if they share an state element between a readSet and a writeSet.
The constraint only actually exists when guards for 
the read and the write of the state element are true.
\begin{dgroup*}
\begin{dmath*}
V \equiv \{all rules, all methods\}
\end{dmath*}
\begin{dmath*}
{ \forall R1, R2 \in V, R1 \neq R2:}\\
        arcCond(R1,R2) \equiv \ORop_{
            \substack{ \forall S \in readSet(R1), \\ S \in writeSet(R2) } }
{
\left(
\begin{split}
guard(R1) \land guard(readSet(S, R1)) \\
\land guard(R2) \land guard(writeSet(S, R2))
\end{split}
\right)
}
\end{dmath*}
\end{dgroup*}

If the arc condition is not identically false, then the
ordering constraint is added, defining a directed graph.
\[\forall R1, R2 \in V, \neg(arcCond(R1, R2) \rightarrow \perp): R1 < R2\]
The set of concurrent transactions is SC iff this graph has no cycles
for any guard conditions.

When detecting cycles in the graph, we calculate the boolean condition
that the entire path of constraints actually occurs and that it is not
identically false:
  \[ pathCondition = \ANDop_{\forall arc(R1,R2) \in path} arcCond(R1, R2) \]
If the pathCondition is not identically false, the compiler breaks cycles by one of the following:
\begin{itemize}
\item Force rules to defer to methods.
if cycle has some method $M$ \& some rule $R$, then the compiler can
automatically rewrite the term $valid(R)$ to add a disjunction with the term $\neg valid(M)$, breaking
the cycle.
\item Fair scheduling between any 2 rules/methods in the cycle.
\end{itemize}

As an optimization, rules and methods that have some overlap of state element usage
(\textit{read set} and \textit{write set})
are greedily partitioned into \textit{schedule sets}
and independently scheduled.

A simple example of a constraint graph is given in \ref{scheduleExample}, at the end of this document.

Since AtomicC performs scheduling analysis independantly for each
declared module, external method invocation conflicts in rules cannot be validated.
Schedule processing for external method calls is delayed until the "module group binding"
stage of linking, where separately compiled AtomicC output is combined and
verified
for SC scheduling.  Errors and conflicts detected at this stage must be repaired
in the module source text and recompiled before proceeding.

\rSec1[atomicc.schedprev]{Previous scheduling work}

In Rule Composition\cite{Dave2007}, scheduling
is formulated in terms of rule composition, leading to a succinct discussion
of issues involved, including a concise description of the Esposito and
Performance Guarantees schedulers.  The resulting schedules are quite close
to the user-specified scheduling in AtomicC.
In contrast to AtomicC, the Bluespec kernel language they use
for analysis also has a sequential composition operator, creating rules
that execute for multiple clock cycles.

The Esposito Scheduler\cite{Esposito:Patent,Dave2007},
is the standard scheduler generation algorithm in the Bluespec Compiler.
It uses a heuristic designed to produce a concrete total ordering of rules.

The Performance Guarantees scheduler\cite{Rosenband:PerformanceGuarantees}
was proposed to address issues with intra-cycle data passing.


\rSec1[scheduleExample]{Scheduling example}{}

\rSec2[orderExample.sw]{Source program}

\begin{codeblock}
__interface UserRequest {
    void say(__uint(32) va);
};

__module Order {
    UserRequest                     request;
    __uint(1) running;
    __uint(32) a, outA, outB, offset;
    void request.say(__uint(32) va) if (!running) {
        a = va;
        offset = 1;
        running = 1;
    }
    __rule A if (!__valid(request.say)) {
        outA = a + offset;
        if (running)
            a = a + 1;
    };
    __rule B if (!__valid(request.say)) {
        outB = a + offset;
        if (!running)
            a = 1;
    };
    __rule C if (!__valid(request.say)) {
        offset = offset + 1;
    };
};
\end{codeblock}


\rSec2[orderExample.conflict]{Constraint graph}

\begin{importgraphic}
{Simple ordering example}
{fig:orderschedule}
{orderschedule.pdf}
\end{importgraphic}

Sequentially consistent schedules are:
%%\setlist{nolistsep}
\begin{itemize}[noitemsep,topsep=0pt]
\item when 'running == 1': A -> B -> C
\item when 'running == 0': B -> A -> C
\end{itemize}

\rSec2[orderExample.verilog]{Verilog output}

\lstinputlisting[language=Verilog,basicstyle=\tiny\ttfamily]{examples/Order.v.tex}

\rSec1[atomicc.schedsignal]{Signalling}

AtomicC uses \textbf{valid/ready} \textit{hand-shaking signalling}
\cite{Fletcher2009, AXISpec} to invoke action methods,
giving both the invoker(master) and invokee(slave) the ability to control invocation
execution timing.
The master uses the \textbf{valid} signal
of an action method to show when parameter data is available
and the operation should be performed.
The method invocation succeeds only when
both \textbf{valid} and \textbf{ready} are HIGH in the same clock cycle.
\newline
In TRS notation\cite[p.~22]{Hoe:Thesis}:
\begin{itemize}[label= ]
\item $\pi(M_{i}) \equiv ready(M_{i})\ \wedge valid(M_{i})$.
\end{itemize}
