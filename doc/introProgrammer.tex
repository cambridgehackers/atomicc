\infannex{introProg}{Introduction for Programmers}


\pnum
\rSec1[introProg.sw]{Software}

In software, the core model is the time-multiplexed execution
of software threads by one or more central processing units (CPUs).
Address arithmetic (pointers and indexing) prevents
the compiler from statically determining read/write
storage elements sets for a transaction.
The programmer is responsible
preventing the interleaved execution of
multiple threads accessing a single storage element by
decoration of the code with library calls to dynamically enforce
mutual exclusion (mutex) regions.

In languages like Java, the programmer is able to decorate the
storage element declarations to automate calling of these mutex operations.


\begin{tikzpicture}
    [title/.style={font=\fontsize{18}{18}\color{black!45}},
    server/.style={rectangle, draw, fill=blue!23},
    client/.style={rectangle, draw, fill=green!23},
    dot/.style={circle, fill=black, minimum size=2pt, inner sep=0pt, outer sep=2pt},
    link/.style={latex-latex},
    >=latex]

    \node [server, rotate=90, outer sep=8pt, align=center] (sched) at (5,8.3) {Scheduler};
    \node [server, fill = red!23] (cpu1) at (7.65,8.3) {CPU};
    \node [server, fill = red!23] (cpu2) at ($(cpu1) + (270:1)$){CPU};
    \node [draw=black!50, fit={(sched) (cpu1) (cpu2)}] (back) {};
    \node [anchor=north east] at (back.north east) {Computer/OS};
    \node [client] (c1) at (0,9.5) {Software thread};
    \node [client] (c2) at ($(c1) + (270:1.15)$) {Software thread};
    \node [client] (c3) at ($(c2) + (270:1.15)$) {Software thread};
    \node [dot] (c4) at ($(c3) + (270:1)$) {};
    \node [dot] (c5) at ($(c4) + (270:0.5)$) {};
    \node [dot] (c6) at ($(c5) + (270:0.5)$) {};
    \draw[->] (c1.east) to ([yshift=2ex]sched.north);
    \draw[->] (c2.east) to (sched.north);
    \draw[->] (c3.east) to ([yshift=-2ex]sched.north);
    \draw[->] (sched.south) to (cpu1.west);
    \draw[->] ([yshift=-2ex]sched.south) to (cpu2.west);
\end{tikzpicture}


\rSec1[introProg.hw]{Hardware}

In hardware, the core model is clock-based updates to state elements from
a combinational logic net.

Combinational logic = current output is a boolean combination of current inputs

Sequential logic = combinational logic + memory elements
(also called finite-state machine)

Synchronous logic = sequential logic + clock

\begin{tikzpicture}[auto, thick, >=triangle 45,fill=blue!20]
\tikzset{  
block/.style    = {draw, thick, rectangle, minimum height = 3em, minimum width = 3em},}
\node at (0,0)[block,draw,fill=blue!20] (clogic) {Combinational Logic};
\node[block,below of=clogic,node distance=2cm,fill=blue!20] (memory) {Memory};
\draw[<-] ([yshift=2ex]clogic.west) to node[midway,above] {$Input$} ++(-3,0);
\draw[->] ([yshift=2ex]clogic.east) to node[midway,above] {$Output$} ++(3,0);
\draw[->] ([yshift=-2ex]clogic.east) -| node[midway,right] {$Next State$} ++(1,0)
                                    |- (memory.east);
\draw[->] (memory.west) -| ++(-2.5,0)
                       |- node[midway,left] {$Current State$} ([yshift=-2ex]clogic.west);
\draw[<-] (memory.south) to ++(0,-1) node[below] {$Clock$};
\end{tikzpicture}

From Hoe\cite{Hoe:Thesis}, the Term Rewriting System representation of this is:

$\textbf{s'}$ = if $\pi(\textbf{s})$ then $\delta(\textbf{s})$ else $\textbf{s}$

Since all hardware elements are independent, all valid source lines in the
program text are executed on every cycle.
Access to state elements supports neither pointers nor indexing, allowing the
compiler to statically determine parallel access transaction conflict sets,
allowing the flagging of all combinations where correct operation cannot
be guaranteed.
